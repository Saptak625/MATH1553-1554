\section{Matrix Operations}
\subsection{Properties}
\noindent
Given that A is an \(m \times n\) matrix.

\noindent
\newline
The following are \textbf{facts}:
\begin{enumerate}
    \item Associative: \((AB)C = A(BC)\)
    \item Left Distributive: \(A(B+C) = AB + AC\)
    \item Right Distributive: \((A+B)C = AC + BC\)
    \item Identity: \(I_m A = A I_n\)
\end{enumerate}

\noindent
\newline
The following are \textbf{non-facts}:
\begin{enumerate}
    \item Non-commutative: \(AB \ne BA\)
    \item Non-cancellation: \(AB = AC\), does not imply \(B=C\)
    \item Zero Divisors: \(AB = 0\) does not imply that either \(A=0\) or \(B=0\).
\end{enumerate}

\subsection{Transpose of a Matrix}
\noindent
\(A^T\) is the matrix whose columns are the rows of \(A\).

\noindent
\newline
Properties:

\begin{enumerate}
    \item \((A^T)^T=A\)
    \item \((A+B)^T = A^T + B^T\)
    \item \((rA)^T = rA^T\)
    \item \((AB)^T = B^T A^T\)
\end{enumerate}

\subsection{Matrix Powers}
\begin{equation}
    A^k = AA\dots A
\end{equation}

\noindent
\newline
When \(A\) is a diagonal matrix (non-zero terms only on the main diagonal), the power computation just exponentiates the main diagonal elements.