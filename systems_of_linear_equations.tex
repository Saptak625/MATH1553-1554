\section{Systems of Linear Equations}
\begin{definition}
Multidimensional Vector Spaces
\begin{itemize}
    \item $\Re$: All Real Numbers
    \item $\Re^2$: All \((x, y)\) such that \(x\) and \(y\) are real.
    \item $\Re^3$: All \((x, y, z)\) such that \(x\), \(y\), and \(z\) are real.
    \item $\Re^n$: All \((x_1, x_2, \dots, x_n)\) such that \(x_1, x_2, \dots, x_n\) are real.
\end{itemize}
\end{definition}

\noindent
Consider \(S(x, y, 0)\), the x-y plane in \(\Re^3\).
\begin{itemize}
    \item \textbf{Question: Is S equal to \(\Re^3\)?}
    \item \textit{Answer: No. \(\Re^2\) has two coordinates. \(\Re^3\) has three coordinates.}
\end{itemize}

\subsection{Lines and Planes in \(\Re^n\)}
\textbf{Ex. 1:}
\(x + y = 1\) in \(\Re^2\)

\noindent
Parameterization:
\(x, 1-x\)

\noindent
\newline
\textbf{Ex. 2:}
\(3x - y + z = 3\) in \(\Re^3\)

\noindent
Parameterization:
\((x, y, 3 - 3x + y)\)

\noindent
\newline
\textbf{Ex. 3:}
\(x - 2y + z - 5w = 7\) in \(\Re^4\) aka "3-plane" in \(Re^4\)

\subsection{Linear Equations}
Examples of Linear Equations:
\begin{align}
    x +2y + 5z &= 3 \\
    x &= 5 - w \\
    2x_1 - 3 x_3 &= \pi + x_2 - ln(n)x_4
\end{align}

\noindent
Examples of Nonlinear Equations:
\begin{align}
    \sqrt{x} \\
    x^2 \\
    sin(x) \\
    ln(y) \\
    xy = 5
\end{align}

\subsubsection{Solution of Linear Equation}
The solution is the intersection of lines, planes, etc.
\begin{definition}
    A linear equation system is considered to be \textbf{consistent} if it has at least one solution. Otherwise, the system is considered to be \textbf{inconsistent}.
\end{definition}

\noindent
\newline
\textbf{Ex. 1: Consistent System. Solution is \((2, 0, 3)\).}
\begin{align}
    x - y + z &= 5 \\
    -x + y + 2z &= 4
\end{align}

\noindent
\newline
\textbf{Ex. 2: Inconsistent Linear Equations in (x, y)}
\begin{align}
    x + y &= 0 \\
    x + y &= 1 
\end{align}

\noindent
\newline
\textbf{Ex. 3: Consistent Linear Equations in (x, y). Solution is at \((3, 1).\)}
\begin{align}
    x - y &= 2 \\
    x + y &= 4
\end{align}

\noindent
\newline
\textbf{Ex. 4: Consistent Linear Equations in (x, y). Infinite number of solutions.}
\begin{align}
    x + y &= 0 \\
    3x + 3y &= 0
\end{align}

\subsection{Augmented Matrices}
\textbf{Ex. 1:}
\begin{align}
    x - y &= 10 \\
    3x + 6y &= 29 
\end{align}

\noindent
\newline
Augemented Matrix Form:
\begin{equation}
    \begin{pmatrix}[cc|c]
      1 & -1 & 10 \\
      3 & 6 & 29
    \end{pmatrix}
\end{equation}

\noindent
\newline
\textit{Solving Process using Row Reduction (Gaussian Elimination):}
\begin{enumerate}
    \item \(R_2 - 3R_1 \rightarrow R_2\)
    \item \(\frac{1}{9}R_2 \rightarrow R_2\)
\end{enumerate}

\begin{align}
    \begin{pmatrix}[cc|c]
      1 & -1 & 10 \\
      0 & 9 & -1
    \end{pmatrix} \\
    \begin{pmatrix}[cc|c]
      1 & -1 & 10 \\
      0 & 1 & -\frac{1}{9}
    \end{pmatrix}
\end{align}
\newline

\noindent
\newline
Back-substitution after Gaussian Elimination:
\begin{align}
    x-y &= 10 \\
    y = -\frac{1}{9}
\end{align}

\noindent
\(\therefore x=\frac{89}{9}\) and \(y=-\frac{1}{9}\)

\noindent
\newline
\textbf{OR:} Gauss-Jordan Elimination:
\begin{align}
    \begin{pmatrix}[cc|c]
      1 & 0 & \frac{89}{9} \\
      0 & 1 & -\frac{1}{9}
    \end{pmatrix}
\end{align}
\newline
\noindent
\(\therefore x=\frac{89}{9}\) and \(y=-\frac{1}{9}\)


\noindent
\textbf{Ex. 2:}
\begin{align}
    x + 2y + 3z &= 6 \\
    2x -3y + 2z &=14 \\
    3x + y -z &= -2
\end{align}

\noindent
\newline
Augemented Matrix Form:
\begin{equation}
    \begin{pmatrix}[ccc|c]
      1 & 2 & 3 & 6 \\
      2 & -3 & 2 & 14 \\
      3 & 1 & -1 & -2 
    \end{pmatrix}
\end{equation}

\noindent
\newline
\textit{Solving Process using Row Reduction:}
\begin{enumerate}
    \item \(R_2 - 2R_1 \rightarrow R_2\)
    \item \(R_3 - 3R_1 \rightarrow R_3\)
    \item \(R_2 \leftrightarrow R_3\)
    \item \(-\frac{1}{5}R_2 \rightarrow R_2\)
    \item \(R3 + 7R_2 \rightarrow R_3\)
    \item \(\frac{1}{10}R_3 \rightarrow R_3\)
\end{enumerate}

\begin{align}
    \begin{pmatrix}[ccc|c]
      1 & 2 & 3 & 6 \\
      0 & -7 & -4 & 2 \\
      3 & 1 & -1 & -2 
    \end{pmatrix} \\
    \begin{pmatrix}[ccc|c]
      1 & 2 & 3 & 6 \\
      0 & -7 & -4 & 2 \\
      0 & -5 & -10 & -20 
    \end{pmatrix} \\
    \begin{pmatrix}[ccc|c]
      1 & 2 & 3 & 6 \\
      0 & -5 & -10 & -20 \\
      0 & -7 & -4 & 2
    \end{pmatrix} \\
    \begin{pmatrix}[ccc|c]
      1 & 2 & 3 & 6 \\
      0 & 1 & 2 & 4 \\
      0 & -7 & -4 & 2
    \end{pmatrix} \\
    \begin{pmatrix}[ccc|c]
      1 & 2 & 3 & 6 \\
      0 & 1 & 2 & 4 \\
      0 & 0 & 10 & 30
    \end{pmatrix} \\
    \begin{pmatrix}[ccc|c]
      1 & 2 & 3 & 6 \\
      0 & 1 & 2 & 4 \\
      0 & 0 & 1 & 3
    \end{pmatrix}
\end{align}

\noindent
\newline
Back-substitution after Gaussian Elimination:
\begin{align}
    x + 2y + 3z &= 6 \\
    y + 2z &= 4 \\
    z &= 3
\end{align}

\noindent
\(\therefore x=1\), \(y=-2\), \(z = 3\)

\begin{enumerate}
    \item \(R_2 - 2R_3 \rightarrow R_2\)
    \item \(R_1 - 3R_3 \rightarrow R_1\)
    \item \(R_1 - 2R_2 \rightarrow R_1\)
\end{enumerate}

\noindent
\newline
\textbf{OR:} Gauss-Jordan Elimination:
\begin{align}
    \begin{pmatrix}[ccc|c]
      1 & 2 & 3 & 6 \\
      0 & 1 & 0 & -2 \\
      0 & 0 & 1 & 3
    \end{pmatrix} \\
    \begin{pmatrix}[ccc|c]
      1 & 2 & 0 & -3 \\
      0 & 1 & 0 & -2 \\
      0 & 0 & 1 & 3
    \end{pmatrix} \\
    \begin{pmatrix}[ccc|c]
      1 & 0 & 0 & 1 \\
      0 & 1 & 0 & -2 \\
      0 & 0 & 1 & 3
    \end{pmatrix}
\end{align}
\newline
\noindent
\(\therefore x=1\), \(y=-2\), \(z = 3\)

\begin{definition}
    Row-Echelon Form:
    \begin{enumerate}
        \item All zero rows (if any) are at the bottom.
        \item Each first non-zero entry (aka leading entry or pivot point) in a row is to the right of the first non-zero entry in the above row.
        \item Below any leading entry, all entries are zero.
    \end{enumerate}
    \[
        \begin{pmatrix}[ccc|c]
            a_1 & a_2 & a_3 & a_4 \\
            0 & a_5 & a_6 & a_7 \\
            0 & 0 & a_8 & a_9
        \end{pmatrix}
    \]
\end{definition}

\begin{definition}
    Reduced Row-Echelon:
    \begin{enumerate}
        \item All row-echelon form requirements must be satisfied.
        \item Each leading entry is 1.
        \item Each pivot point is the \textbf{only} non-zero entry in its column.
    \end{enumerate}
     \[
        \begin{pmatrix}[ccc|c]
            1 & 0 & 0 & a_1 \\
            0 & 1 & 0 & a_2 \\
            0 & 0 & 1 & a_3
        \end{pmatrix}
    \]
\end{definition}

\noindent
\textbf{Ex. 3:}
\begin{align}
    2x_1 - 3x_2 + x_3 + x_4 &= 6 \\
    x_1 - x_2 - x_4 &= 4 \\
    x_2 - x_3 &= 2
\end{align}

\noindent
\newline
Augemented Matrix Form:
\begin{equation}
    \begin{pmatrix}[cccc|c]
      2 & -3 & 1 & 1 & 6 \\
      1 & -1 & 0 & -1  & 4 \\
      0 & 1 & -1 & 0 & 2
    \end{pmatrix}
\end{equation}

\noindent
\newline
\textit{Solving Process using Row Reduction:}
\begin{enumerate}
    \item \(R_1 \leftrightarrow R_2\)
    \item \(R_2 - 2R_1 \rightarrow R_2\)
    \item \(-R_2 \rightarrow R_2\)
    \item \(R_3 - R_2 \rightarrow R_3\)
    \item \(\frac{1}{3} R_3 \rightarrow R_3\)
    \item \(3R_3 + R_2 \rightarrow R_2\)
    \item \(R_3 + R_1 \rightarrow R_1\)
    \item \(R_2 + R_1 \rightarrow R_1\)
\end{enumerate}

\begin{align}
    \begin{pmatrix}[cccc|c]
      1 & -1 & 0 & -1  & 4 \\
      2 & -3 & 1 & 1 & 6 \\
      0 & 1 & -1 & 0 & 2
    \end{pmatrix} \\
    \begin{pmatrix}[cccc|c]
      1 & -1 & 0 & -1  & 4 \\
      0 & -1 & 1 & 3 & -2 \\
      0 & 1 & -1 & 0 & 2
    \end{pmatrix} \\
    \begin{pmatrix}[cccc|c]
      1 & -1 & 0 & -1  & 4 \\
      0 & -1 & 1 & 3 & -2 \\
      0 & 1 & -1 & 0 & 2
    \end{pmatrix} \\
    \begin{pmatrix}[cccc|c]
      1 & -1 & 0 & -1  & 4 \\
      0 & 1 & -1 & -3 & 2 \\
      0 & 1 & -1 & 0 & 2
    \end{pmatrix} \\
    \begin{pmatrix}[cccc|c]
      1 & -1 & 0 & -1  & 4 \\
      0 & 1 & -1 & -3 & 2 \\
      0 & 0 & 0 & 3 & 0
    \end{pmatrix} \\
    \begin{pmatrix}[cccc|c]
      1 & -1 & 0 & -1  & 4 \\
      0 & 1 & -1 & -3 & 2 \\
      0 & 0 & 0 & 1 & 0
    \end{pmatrix} \\
    \begin{pmatrix}[cccc|c]
      1 & -1 & 0 & -1  & 4 \\
      0 & 1 & -1 & 0 & 2 \\
      0 & 0 & 0 & 1 & 0
    \end{pmatrix} \\
    \begin{pmatrix}[cccc|c]
      1 & -1 & 0 & 0  & 4 \\
      0 & 1 & -1 & 0 & 2 \\
      0 & 0 & 0 & 1 & 0
    \end{pmatrix} \\
\end{align}

\subsection{Parametric Forms}
\textbf{Write the solution set to \(x+y+z=10\).}
\begin{equation}
    \begin{pmatrix}[ccc|c]
        1 & 1 & 1 & 10
    \end{pmatrix}
\end{equation}

\noindent
\newline
The first one is a pivot. The other ones are free variables.

\begin{align}
    x &= 10 - y - z \\
    y &= y \\
    z &= z
\end{align}

\noindent
\newline
Geometrically, the system is a plane in \(\Re^3\).