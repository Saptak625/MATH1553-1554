\section{Quadratic Forms}
\subsection{Definition}
\begin{definition}
    A quadratic form is a function \(Q: \Re^n \rightarrow \Re\), given by
    \[Q(\Vec{x}) = \Vec{x}^T A \Vec{x}\]
\end{definition}

\subsection{Converting between Quadratic Forms}
\textbf{Example 1:}
\begin{align}
    Q_A(x_1, x_2) = \Vec{x}^T A \Vec{x} &= \begin{pmatrix}
        x_1 x_2
    \end{pmatrix} \begin{pmatrix}
        4 & 0 \\ 0 & 3
    \end{pmatrix} \begin{pmatrix}
        x_1 \\ x_2
    \end{pmatrix} \\ &= 4x_1^2 + 3x_2^2 \\
    Q_B(x_1, x_2) = \Vec{x}^T A \Vec{x} &= \begin{pmatrix}
        x_1 x_2
    \end{pmatrix} \begin{pmatrix}
        4 & 1 \\ 1 & -3
    \end{pmatrix} \begin{pmatrix}
        x_1 \\ x_2
    \end{pmatrix} \\ &= 4x_1^2 + 2x_1 x_2 - 3x_2^2
\end{align}

\noindent
\newline
\textbf{Example 2: Write \(Q\) in the form \(\Vec{x}^T A \Vec{x}\).}
\begin{align}
    Q_A(x_1, x_2) = \Vec{x}^T A \Vec{x} &= 5x_1^2 - x_2^2 + 3x_3^2 + 6x_1 X_3 - 12 x_2 x_3 \\
    &= \begin{pmatrix}
        x_1 x_2 x_3
    \end{pmatrix} \begin{pmatrix}
        5 & 0 & 3 \\
        0 & -1 & -6 \\
        3 & -6 & 3
    \end{pmatrix} \begin{pmatrix}
        x_1 \\ x_2 \\ x_3
    \end{pmatrix} \\
\end{align}

\subsection{Change of Variables}
\textbf{Procedure to find \(\Vec{y} = P^T \Vec{x}\):}
\begin{align}
    A &= PDP^T \\
    Q_A(\Vec{x}) &= \Vec{x}^T A \Vec{x} \\
    &= \Vec{x}^T PDP^T \Vec{x} \\
    &= (P^T \Vec{x})^T D P^T \Vec{x} \\
    &= \Vec{y}^T D \Vec{y} \\
    &= Q_D(\Vec{y})
\end{align}
In order to remove cross-terms from the quadratic form, one can define \(Q_A(\Vec{x}) = Q_D(\Vec{y}) = \Vec{y}^T D \Vec{y}\). \(P^T\) is a rotation matrix that transforms the quadratic form such that the cross-terms when written as a function of \(\Vec{x}\) are removed when written as a function of \(\Vec{y}=P^T \Vec{x}\).

\noindent
\newline
\textbf{Example 1:}
\begin{align}
    A &= \begin{pmatrix}
        3 & 2 \\ 2 & 6
    \end{pmatrix} = PDP^T \\
    P &= \frac{1}{\sqrt{5}} \begin{pmatrix}
        2 & 1 \\ -1 & 2
    \end{pmatrix} \\
    D &= \begin{pmatrix}
        2 & 0 \\ 0 & 7
    \end{pmatrix} \\
    Q_A(x_1, x_2) &= 3x_1^2 + 4x_1 x_2 + 6x_2^2 \\
    Q_D(y_1, y_2) &= 2y_1^2 + 7y_2^2 \\
    \Vec{y} &= P^T \Vec{x} \\
    \begin{pmatrix}
        y_1 \\ y_2
    \end{pmatrix} &= \begin{pmatrix}
        \frac{2}{\sqrt{5}} x_1 -\frac{1}{\sqrt{5}} x_2 \\
        \frac{1}{\sqrt{5}} x_1 +\frac{2}{\sqrt{5}} x_2
    \end{pmatrix}
\end{align} 

\begin{theorem} Principal Axes Theorem

    If \(A\) is a symmetric matrix then there exists an orthogonal change of variable \(\Vec{x}=P\Vec{y}\) that transforms \(\Vec{x}^T A \Vec{x}\) to \(\Vec{y}^T D \Vec{y}\) with no cross-product terms.
\end{theorem}

\subsection{Classifying Quadratic Forms}
\begin{definition}
    A quadratic form \(Q\) is:
    \begin{enumerate}
        \item positive definite if \(Q(\Vec{x}) > 0\) for all \(\Vec{x}\ne\Vec{0}\).
        \item negative definite if \(Q(\Vec{x}) < 0\) for all \(\Vec{x}\ne\Vec{0}\).
        \item positive semidefinite if \(Q(\Vec{x}) \ge 0\) for all \(\Vec{x}\).
        \item negative semidefinite if \(Q(\Vec{x}) \le 0\) for all \(\Vec{x}\).
        \item indefinite if \(Q(\Vec{x}) \in \Re\) for all \(\Vec{x}\).
    \end{enumerate}
\end{definition}

\subsection{Quadratic Forms and Eigenvalues}
\begin{theorem}
    If \(A\) is a symmetric matrix with eigenvalues \(\lambda_i\), then \(Q = \Vec{x}^T A \Vec{x}\) is:
    \begin{enumerate}
        \item positive definite iff \(\lambda_i > 0\).
        \item negative definite iff \(\lambda_i < 0\).
        \item positive semidefinite iff \(\lambda_i \ge 0\).
        \item negative semidefinite iff \(\lambda_i \le 0\).
        \item indefinite iff \(\lambda_i > 0\) and some other \(\lambda_i < 0\).
    \end{enumerate}
\end{theorem}