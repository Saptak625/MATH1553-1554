\section{Linear Transformations}
\subsection{Introduction}
Functions can be expressed as so:
\begin{equation}
    f: \; \Re \rightarrow \Re
\end{equation}

\noindent
A \(m \times n\) matrix transformation can be expressed as:
\begin{equation}
    T: \; \Re^n \rightarrow \Re^m, \; T(\vec{v}) = A \vec{v}
\end{equation}

\noindent
The vector \(T(\vec{v})\) is the image of \(\vec{v}\) under \(T\). The set of all possible images \(T(\vec{v})\) is the range.

\begin{definition}
    A function \(T\):  \(\Re^n\rightarrow\Re^m\) is linear if:
    \begin{itemize}
        \item \(T(\vec{u}+\vec{v}) = T(\vec{u}) + T(\vec{v})\) for all \(\vec{u}, \vec{v}\) in \(\Re^n\).
        \item \(T(c\vec{v}) = cT(\vec{v})\) for all \(\vec{v}\) in \(\Re^n\), and \(c\) in \(\Re\).
    \end{itemize}
\end{definition}

\begin{definition}
    The principle of superposition:
    
    If \(T\) is linear, then:
    \[T(c_1\Vec{v_1} + \dots + c_k \Vec{v_k}) = c_1 T(\vec{v_1}) + \dots + c_k T(\vec{v_k})\]
\end{definition}

\subsection{Standard Basis Vectors}
\noindent
For example, in \(\Re^3\),
\begin{align}
    \Vec{e_1} &= \begin{pmatrix}
        1 \\ 0 \\ 0
    \end{pmatrix} \\
    \Vec{e_2} &= \begin{pmatrix}
        0 \\ 1 \\ 0
    \end{pmatrix} \\
    \Vec{e_3} &= \begin{pmatrix}
        0 \\ 0 \\ 1
    \end{pmatrix}
\end{align}

\noindent
\newline
What is the linear transform \(T(\Vec{x})=A\Vec{x}\) to rotate counterclockwise by angle \(\theta\)?

\begin{align}
    A &= \begin{pmatrix}
        cos(\theta) & cos(\theta + \frac{\pi}{2}) \\
        sin(\theta) & sin(\theta + \frac{\pi}{2}) \\
    \end{pmatrix} \\
    &= \begin{pmatrix}
        cos(\theta) & -sin(\theta) \\
        sin(\theta) & cos(\theta) \\
    \end{pmatrix}
\end{align}

\noindent
\newline
Let \(T(\Vec{x}=A\Vec{x}\) be the transformation which first reflects vectors in \(\Re^2\) across the line \(y=0\) and then projects the resulting vector to the y-axis. Thus:
\begin{equation}
    A = \begin{pmatrix}
        0 & 0 \\
        0 & -1
    \end{pmatrix}
\end{equation}

\subsection{Onto and One-to-one Transformations}
\begin{definition}
    A linear transformation \(T: \; \Re^n \rightarrow \Re^m\) is onto if for all \(\Vec{b} \in \Re^m\) there is a \(\Vec{x} \in \Re^n\) so that \(T(\Vec{x})=\Vec{b}\). In other words, \(A\Vec{x}=\Vec{b}\) is always consistent. There are pivots in every row.
\end{definition}

\begin{definition}
    A linear transformation \(T: \; \Re^n \rightarrow \Re^m\) is one-to-one if for all \(\Vec{b} \in \Re^m\) there is at most one (possibly no) so that \(T(\Vec{x})=\Vec{b}\). There are pivots in every column.
\end{definition}

\noindent
\newline
\textbf{One-to-one but not onto?}

\noindent
Tall matrix. There is a pivot in every column, but not a pivot in every row.

\noindent
\newline
\textbf{Onto but not one-to-one?}

\noindent
Wide matrix. There is a pivot in every row, but not in every column.
