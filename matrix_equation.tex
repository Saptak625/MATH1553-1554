\section{Matrix Equation}
\subsection{Linear Combinations and Span}
\begin{definition}
    Linear Combinations
    \newline
    A is a \(m\times n\) matrix with columns \(\Vec{a_1}, \dots, \Vec{a_n}\) and \(x\in \Re^n\), then the matrix-vector product \(A\Vec{x}\) is a linear combination of the columns of A:
    \begin{equation}
        A\Vec{x} = 
        \begin{pmatrix}
            \spike{15pt}{$\Vec{a_1}$}  & \spike{15pt}{$\Vec{a_2}$} & \spike{15pt}{$\Vec{a_3}$}
        \end{pmatrix}
        \begin{pmatrix}
            x_1 \\
            x_2 \\
            x_3
        \end{pmatrix} = x_1 \Vec{a_1} + x_2 \Vec{a_2} + x_3 \Vec{a_3}
    \end{equation}
\end{definition}

\begin{definition}
    Span 
    \newline
    The Span of a set of vectors \(\Vec{v_1}, \dots, \Vec{v_p}\) in \(Re^n\) is: \[\text{Span}\{\Vec{v_1}, \dots, \Vec{v_p}\} = \{x_1 \Vec{v_1} + \dots + x_2 \Vec{v_p} \; | \; x_1, \dots, x_p \in \Re\}\]
\end{definition}

\noindent
For what vectors \(\Vec{b} = \begin{pmatrix}
    b_1 \\ b_2 \\ b_3
\end{pmatrix}\), does the equation have a solution?

\begin{equation}
    \begin{pmatrix}
        1 & 3 & 4 \\
        2 & 8 & 4 \\
        0 & 1 & -2
    \end{pmatrix}
    \Vec{x} = \Vec{b}
\end{equation}

\noindent
\textbf{Solution: Row Reduce \([A | b]\)}
\begin{align}
    \begin{pmatrix}[ccc|c]
        1 & 3 & 4 & b_1 \\
        2 & 8 & 4 & b_2 \\
        0 & 1 & -2 & b_3
    \end{pmatrix} \\
    \begin{pmatrix}[ccc|c]
        1 & 3 & 4 & b_1 \\
        0 & 2 & -4 & -2b_1 + b_2 \\
        0 & 1 & -2 & b_3
    \end{pmatrix} \\
    \begin{pmatrix}[ccc|c]
        1 & 3 & 4 & b_1 \\
        0 & 1 & -2 & b_3 \\
        0 & 2 & -4 & -2b_1 + b_2
    \end{pmatrix} \\
    \begin{pmatrix}[ccc|c]
        1 & 3 & 4 & b_1 \\
        0 & 1 & -2 & b_3 \\
        0 & 0 & 0 & -2b_1 + b_2 - 2b_3
    \end{pmatrix}
\end{align}

\noindent
\(-2b_1 + b_2 - 2b_3 \neq 0\) is a pivot. Furthermore, the system will be consistent when \(-2b_1 + b_2 - 2b_3 = 0\).

\noindent
\newline
\textbf{Q: When is \(\Vec{b} = \begin{pmatrix}
    b_1 \\ b_2 \\ b_3
\end{pmatrix}\) in the span of \(\begin{pmatrix}
    1 \\ 2 \\ 0
\end{pmatrix}\), \(\begin{pmatrix}
    3 \\ 8 \\ 1
\end{pmatrix}\), \(\begin{pmatrix}
    4 \\ 4 \\ -2
\end{pmatrix}\)?}

\noindent
Same solution. \(\Vec{b}\) is in the span of the vectors when it can be expressed as a linear combination. In other words, some real scalars \(x_1, x_2, x_3\) can be used to create a linear combination of the three column vectors to result in \(\Vec{b}\).

\noindent
\newline
\textbf{Let A = \(\begin{pmatrix}
    1 & 3 & 4 \\
    -4 & 2 & -6 \\
    -3 & -2 & -7
    \end{pmatrix}\) and \(\Vec{b} = \begin{pmatrix}
        b_1 \\ b_2 \\ b_3
    \end{pmatrix}\). Is the equation \(A\Vec{x} = \Vec{b}\) consistent for all possible \(b_1, b_2, b_3\)?}

\noindent
Solution: Row Reduce A
\begin{align}
    \begin{pmatrix}
    1 & 3 & 4 \\
    -4 & 2 & -6 \\
    -3 & -2 & -7
    \end{pmatrix} \\
    \begin{pmatrix}
        1 & 3 & 4 \\
        0 & 7 & 5 \\
        0 & 0 & 0
    \end{pmatrix} \\
\end{align}

\noindent
If A does not have a pivot in every row, then some choice of \(\Vec{b}\) will make a pivot in \([A|b]\) inconsistent.

\subsection{Solution Sets of Linear Systems}
\begin{definition}
    Homogeneity
    Linear systems of the form \(A\Vec{x} = \Vec{0}\) are homogeneous.
    Linear systems of the form \(A\Vec{x} = \Vec{b}\) are inhomogeneous.
\end{definition}

\noindent
Because homogeneous systems always have the trivial solution, \(\Vec{x} = \Vec{0}\), the interesting question is how many solutions the system will have.

\noindent
\newline
\textbf{Solution Set for a Homogeneous System:}

\noindent
\textit{Row Reduce \([A|\Vec{0}]\)}

\begin{align}
    \begin{pmatrix}[ccc|c]
        1 & 3 & 1 & 0 \\
        2 & -1 & 5 & 0 \\
        1 & 0 & -2 & 0
    \end{pmatrix} \\
    \begin{pmatrix}[ccc|c]
        1 & 3 & 1 & 0 \\
        0 & -7 & -7 & 0 \\
        0 & -3 & -3 & 0
    \end{pmatrix} \\
    \begin{pmatrix}[ccc|c]
        1 & 3 & 1 & 0 \\
        0 & 1 & 1 & 0 \\
        0 & 1 & 1 & 0
    \end{pmatrix} \\
    \begin{pmatrix}[ccc|c]
        1 & 3 & 1 & 0 \\
        0 & 1 & 1 & 0 \\
        0 & 0 & 0 & 0
    \end{pmatrix} \\
    \begin{pmatrix}[ccc|c]
        1 & 0 & -2 & 0 \\
        0 & 1 & 1 & 0 \\
        0 & 0 & 0 & 0
    \end{pmatrix}
\end{align}

\noindent
\newline
Rewrite the Equations:
\begin{align}
    x_1 - 2x_3 &= 0 \\
    x_2 + x_3 &= 0 \\
    x_3 &= x_3
\end{align}

\noindent
\newline
Move free variables to RHS. This results in parametric equation form.
\begin{align}
    x_1 &= 2x_3 \\
    x_2 &= -x_3 \\
    x_3 &= x_3
\end{align}

\noindent
\newline
Therefore, the vector parametric form is:
\begin{equation}
    \Vec{x} = x_3 \begin{pmatrix}
        2 \\
        -1 \\
        1
    \end{pmatrix}
\end{equation}

\noindent
The solution is a line in \(\Re^3\).

\noindent
\newline
\textbf{Inhomogeneous Case:}
\begin{align}
    \begin{pmatrix}[ccc|c]
        1 & 3 & 1 & 9 \\
        2 & -1 & 5 & 11 \\
        1 & 0 & -2 & 6
    \end{pmatrix} \\
    \begin{pmatrix}[ccc|c]
        1 & 3 & 1 & 9 \\
        0 & -7 & -7 & -7 \\
        0 & -3 & -3 & -3
    \end{pmatrix} \\
    \begin{pmatrix}[ccc|c]
        1 & 3 & 1 & 9 \\
        0 & 1 & 1 & 1 \\
        0 & 1 & 1 & 1
    \end{pmatrix} \\
    \begin{pmatrix}[ccc|c]
        1 & 0 & -2 & 6 \\
        0 & 1 & 1 & 1 \\
        0 & 0 & 0 & 0
    \end{pmatrix}
\end{align}

\noindent
\newline
Rewrite the Equations:
\begin{align}
    x_1 - 2x_3 &= 6 \\
    x_2 + x_3 &= 1 \\
    x_3 &= x_3
\end{align}

\noindent
\newline
Move free variables to RHS. This results in parametric equation form.
\begin{align}
    x_1 &= 6 + 2x_3 \\
    x_2 &= 1 - x_3 \\
    x_3 &= x_3
\end{align}

\noindent
\newline
Therefore, the vector parametric form is:
\begin{equation}
    \Vec{x} = \begin{pmatrix}
        6 \\
        1 \\
        0
    \end{pmatrix} + x_3 \begin{pmatrix}
        2 \\
        -1 \\
        1
    \end{pmatrix}
\end{equation}

\noindent
The solution is a line in \(\Re^3\). The particular solution is \(\begin{pmatrix}
        6 \\
        1 \\
        0
\end{pmatrix}\). The general solutions to \(A\Vec{x}=\Vec{b}\) are general solutions of \(A\Vec{x}=\Vec{0}\) plus a particular solution.

\noindent
\newline
\textbf{Find general solutions to both \(A\Vec{x}=\Vec{b}\) and \(A\Vec{x}=\Vec{0}\).}
\begin{align}
    A&=\begin{pmatrix}
        1 & 3 & 0 & -4 \\
        2 & 6 & 0 & -8
    \end{pmatrix} \\
    \Vec{x} &= \begin{pmatrix}
        1 \\ 2 \\ 3 \\ 4
    \end{pmatrix} \\
    \Vec{b} &= \begin{pmatrix}
        -9 \\ -18
    \end{pmatrix}
\end{align}

\noindent
\newline
Row reduce \([A | b]\).
\begin{align}
    \begin{pmatrix}[cccc|c]
        1 & 3 & 0 & -4 & -9 \\
        2 & 6 & 0 & -8 & -18
    \end{pmatrix} \\
    \begin{pmatrix}[cccc|c]
        1 & 3 & 0 & -4 & -9 \\
        0 & 0 & 0 & 0 & 0
    \end{pmatrix}
\end{align}

\noindent
\newline
Rewrite the Equations:
\begin{align}
    x_1 + 3x_2 -4x_4 &= -9 \\
    x_2 &= x_2 \\
    x_3 &= x_3 \\
    x_4 &= x_4 \\
\end{align}

\noindent
\newline
Move free variables to RHS. This results in parametric equation form.
\begin{align}
    x_1 &= -9 - 3x_2 + 4 x_4 \\
    x_2 &= x_2 \\
    x_3 &= x_3 \\
    x_4 &= x_4 \\
\end{align}

\noindent
\newline
Therefore, the vector parametric form is:
\begin{equation}
    \Vec{x} = \begin{pmatrix}
        -9 \\
        0 \\
        0 \\
        0
    \end{pmatrix} + x_2 \begin{pmatrix}
        -3 \\
        1 \\
        0 \\
        0
    \end{pmatrix} + x_3 \begin{pmatrix}
        0 \\
        0 \\
        1 \\
        0
    \end{pmatrix} + x_4 \begin{pmatrix}
        4 \\
        0 \\
        0 \\
        1
    \end{pmatrix}
\end{equation}
The solution is a 3D-object in \(\Re^4\). The particular solution is \(\begin{pmatrix}
        -9 \\
        0 \\
        0 \\
        0
\end{pmatrix}\). The general solutions to \(A\Vec{x}=\Vec{b}\) are general solutions of \(A\Vec{x}=\Vec{0}\) plus a particular solution. Therefore, the solution to \(A\Vec{x} = \Vec{0}\) is:
\begin{equation}
    \Vec{x} = x_2 \begin{pmatrix}
        -3 \\
        1 \\
        0 \\
        0
    \end{pmatrix} + x_3 \begin{pmatrix}
        0 \\
        0 \\
        1 \\
        0
    \end{pmatrix} + x_4 \begin{pmatrix}
        4 \\
        0 \\
        0 \\
        1
    \end{pmatrix}
\end{equation}

\subsection{Linear Independence}
\begin{definition}
    An indexed set of vectors \(\{\Vec{v_1}, \dots, \Vec{v_p}\}\) in \(Re^n\) is said to be linearly independent if the vector equation:
    \[x_1\Vec{v_1} + x_2\Vec{v_2} + \dots + x_p \Vec{v_p} = 0\]
    has only the trivial solution. The set \(\{\Vec{v_1}, \dots, \Vec{v_p}\}\) is said to be linearly dependent if the vector equation \[x_1\Vec{v_1} + x_2\Vec{v_2} + \dots + x_p \Vec{v_p} = 0\] has an infinite number of solutions where some values in \(x_1, \dots, x_p\) are non-zero.
\end{definition}

\noindent
\newline
\textbf{Ex. 1: Is the following linearly independent: \(\begin{pmatrix}1 \\ 0 \\ 1\end{pmatrix}, \begin{pmatrix}2 \\ 1 \\ 3\end{pmatrix}, \begin{pmatrix}0 \\ 1 \\ 1\end{pmatrix}\)}

\noindent
\newline
Solution:
\begin{align}
    \begin{pmatrix}
        1 & 2 & 0 \\
        0 & 1 & 1 \\
        1 & 3 & 1
    \end{pmatrix} \\
    \begin{pmatrix}
        1 & 2 & 0 \\
        0 & 1 & 1 \\
        0 & 1 & 1
    \end{pmatrix} \\
    \begin{pmatrix}
        1 & 0 & -2 \\
        0 & 1 & 1 \\
        0 & 0 & 0
    \end{pmatrix} \\
\end{align}

\noindent
\newline
Rewrite the Equations:
\begin{align}
    x_1 - 2x_3 &= 0 \\
    x_2 + x_3 &= 0 \\
    x_3 &= x_3
\end{align}

\noindent
\newline
Move free variables to RHS. This results in parametric vector form.
\begin{equation}
    \Vec{x} = x_3 \begin{pmatrix}
        2 \\
        -1 \\
        1
    \end{pmatrix}
\end{equation}

\noindent
\newline
Therefore, the system is linearly dependent because there are non-trivial solutions to the coefficient matrix. Specifically, one dependence relation is:
\begin{equation}
    \begin{pmatrix}
        1 & 2 & 0 \\
        0 & 1 & 1 \\
        1 & 3 & 1
    \end{pmatrix} \begin{pmatrix}
        2 \\
        -1 \\
        1
    \end{pmatrix} = 2\begin{pmatrix}1 \\ 0 \\ 1\end{pmatrix} - \begin{pmatrix}2 \\ 1 \\ 3\end{pmatrix} + \begin{pmatrix}0 \\ 1 \\ 1\end{pmatrix} = \begin{pmatrix}0 \\ 0 \\ 0\end{pmatrix}
\end{equation}

\noindent
\newline
\textbf{Ex. 2: Is the following linearly independent: \(\begin{pmatrix}1 \\ 0 \\ 1\end{pmatrix}, \begin{pmatrix}2 \\ 2 \\ 0\end{pmatrix}, \begin{pmatrix}0 \\ 2 \\ 1\end{pmatrix}\)}

\noindent
\newline
Solution:
\begin{align}
    \begin{pmatrix}[ccc|c]
        1 & 2 & 0 & 0\\
        0 & 2 & 2 & 0 \\
        1 & 0 & 1 & 0
    \end{pmatrix} \\
    \begin{pmatrix}
        1 & 2 & 0 & 0\\
        0 & 1 & 1 & 0 \\
        0 & -2 & 1 & 0
    \end{pmatrix} \\
    \begin{pmatrix}
        1 & 0 & 0 & 0\\
        0 & 1 & 0 & 0 \\
        0 & 0 & 1 & 0
    \end{pmatrix} \\
\end{align}

\noindent
\newline
Rewrite the Equations:
\begin{align}
    x_1 &= 0 \\
    x_2 &= 0 \\
    x_3 &= 0
\end{align}

\noindent
\newline
Therefore, the system is linearly independent because there is only the trivial solution to the coefficient matrix.
\begin{equation}
    \begin{pmatrix}
        1 & 2 & 0 \\
        0 & 2 & 2 \\
        1 & 0 & 1
    \end{pmatrix} \begin{pmatrix}
        0 \\
        0 \\
        0
    \end{pmatrix} = 0\begin{pmatrix}1 \\ 0 \\ 1\end{pmatrix} + 0\begin{pmatrix}2 \\ 2 \\ 0\end{pmatrix} + 0\begin{pmatrix}0 \\ 2 \\ 1\end{pmatrix} = \begin{pmatrix}0 \\ 0 \\ 0\end{pmatrix}
\end{equation}

\noindent
\newline
\textbf{Ex. 3: Is the following linearly independent: \(\begin{pmatrix}1 \\ 0 \\ 0\end{pmatrix}, \begin{pmatrix}0 \\ 1 \\ 0\end{pmatrix}, \begin{pmatrix}3 \\ 2 \\ 0\end{pmatrix}\)}

\noindent
\newline
Solution: Linearly dependent. A dependence relation is:
\begin{equation}
    3\begin{pmatrix}1 \\ 0 \\ 0\end{pmatrix} + 2\begin{pmatrix}0 \\ 1 \\ 0\end{pmatrix} - \begin{pmatrix}3 \\ 2 \\ 0\end{pmatrix} = \begin{pmatrix}0 \\ 0 \\ 0\end{pmatrix}
\end{equation}