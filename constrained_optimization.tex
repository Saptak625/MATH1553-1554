\section{Constrained Optimization}
\subsection{Quadratic Forms}
\textbf{Example 1: Find the largest and smallest of \(Q(\Vec{x}) = 9 x_1^2 + 4 x_2^2 + 3 x_3^2 \) on the surface of an unit sphere.}

\begin{align}
    \text{Max: }& Q(1, 0, 0) = 9 \\
    \text{Min: }& Q(0, 0, 1) = 3
\end{align}

\noindent
\newline
If \(Q(\Vec{x}) = a x_1^2 + b x_2^2 + c x_3^2\) and \(||v|| = 1\), then \(Q(\Vec{v}\) is a weighted average of \(a\), \(b\), \(c\).

\subsection{Using Eigenvalues for Constrained Optimization}
\noindent
\newline
\textbf{Example 1: Find the largest and smallest of \(Q(\Vec{x}) = 3 x_1^2 + 7 x_2^2 \) on the edge of an unit circle.}

\begin{align}
    \text{Max: }& Q(0, 1) = 7 \\
    \text{Min: }& Q(1, 0) = 3
\end{align}

\noindent
\newline
\textbf{Example 3: } Let \(A = \begin{pmatrix}
    3 & 2 & 1 \\ 2 & 3 & 1 \\ 1 & 1 & 4
\end{pmatrix}\). Find the maximum value of the quadratic form \(\Vec{x^T A \Vec{x}}\) subject to the constraint \(\Vec{x}^T \Vec{x} = 1\) and find an unit vector at which this maximum value is attained.

\begin{align}
    Q(x_1, x_2, x_3) &= 3 x_1^2 + 3 x_2^2 + 4 x_3^2 + 4 x_1 x_2 + 2 x_1 x_3 + 2 x_2 x_3 \\
    p(\lambda) &= -(\lambda-6)(\lambda-3)(\lambda-1) = 0 \\
    \lambda_1 = 6, \lambda_2 &= 3, \lambda_3 = 1 \\
    Q_D &= 6y_1^2 + 3y_2^2 + y_3^2 \\
    \text{Max: }& Q_D(1, 0, 0) = 6
\end{align}

\begin{align}
    P \Vec{y} &= \Vec{x} \Leftarrow \text{Unit Length Eigenvector in 6-eigenspace} \\
    A-6I = \begin{pmatrix}
        -3 & 2 & 1 \\ 2 & -3 & 1 \\ 1 & 1 & -2
    \end{pmatrix}
    &\rightarrow \begin{pmatrix}
        1 & 1 & -2 \\ 0 & 5 & -5 \\ 0 & -5 & 5
    \end{pmatrix} \rightarrow \begin{pmatrix}
        1 & 0 & -1 \\ 0 & 1 & -1 \\ 0 & 0 & 0
    \end{pmatrix} \\
    \Vec{x} &= \begin{pmatrix}
        \frac{\sqrt{3}}{3} \\ \frac{\sqrt{3}}{3} \\ \frac{\sqrt{3}}{3}
    \end{pmatrix} \\
    \text{Max: }& Q(\Vec{x}) = 6
\end{align}

\subsection{Constrained Optimization and Eigenvalues}
\begin{theorem}
    If \(Q = \Vec{x}^T A \Vec{x}\), \(A\) is a real \(n\times n\) symmetric matrix, with eigenvalues:
    \[\lambda_1 \ge \lambda_2 \dots \ge \lambda_n\]
    and associated normalized eigenvectors:
    \[\Vec{u_1},\Vec{u_2},\dots,\Vec{u_n}\]
    Then, subject to the constraint \(||\Vec{x}|| = 1\),
    \begin{itemize}
        \item the maximum value of \(Q(\Vec{x}) = \lambda_1\), attained at \(\Vec{x}=\pm \Vec{u_1}\).
        \item the minimum value of \(Q(\Vec{x}) = \lambda_n\), attained at \(\Vec{x}=\pm \Vec{u_n}\).
    \end{itemize}
\end{theorem}