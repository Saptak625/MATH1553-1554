\section{Determinants}
\subsection{Definition}
\begin{definition}
    Definition of Determinant
    Suppose \(A\) is \(n \times n\) and has elements \(a_{ij}\).
\begin{enumerate}
    \item If \(n=1\), \(A=a_{11}\), and has determinant \(\text{det} \; A = a_{11}\).
    \item Inductive case: for \(n > 1\),
    \[\text{det} \; A = a_{11} \text{det} \; A_{11} - a_{12} \text{det} \; A_{12} + \dots + (-1)^{1+n} a_{1n} \text{det} \; A_{1n}\]
    where \(A_{ij}\) is the submatrix obtained by eliminating row \(i\) and column \(j\) of \(A\).
\end{enumerate}
\end{definition}

\begin{definition}
    Cofactors

    The \((i, j)\) cofactor of an \(n\times n\) matrix A is:
    \[C_{ij} = (-1)^{i+j} \text{det} \; A_{ij}\]
\end{definition}

\begin{theorem}
    The determinant of a matrix \(A\) can be computed down any row or column of the matrix. For instance, down the jth column, the determinant is:
    \[\text{det} \; A = a_{1j} C_{1j} + a_{2j} C_{2j} + \dots + a_{nj} C_{nj}\]
\end{theorem}

\subsection{Determinant of a \(3 \times 3\) Matrix}
\begin{align}
    \text{det} \begin{pmatrix}
        1 & -5 & 0 \\
        2 & 4 & -1 \\
        0 & 2 & 0
    \end{pmatrix} = 1 \begin{vmatrix}
        4 & -1 \\ 2 & 0
    \end{vmatrix} - (-5) \begin{vmatrix}
        2 & -1 \\ 0 & 0
    \end{vmatrix} + 0 \begin{vmatrix}
        2 & 4 \\ 0 & 2
    \end{vmatrix} = 2
\end{align}

\subsection{Determinants of Triangular Matrices}
\begin{theorem}
    If \(A\) is a triangular matrix then,
    \[\text{det} \; A = a_{11} a_{22} a_{33} \dots a_{nn}\]
\end{theorem}

\subsection{Properties of the Determinant}
\begin{theorem}
    For any square matrices \(A\) and \(B\), we can show the following.
    \begin{enumerate}
        \item \(\text{det} \; A = \text{det} \; A^T\)
        \item \(A\) is invertible iff \(\text{det} \; A \ne 0\).
        \item \(\text{det} (A B)= (\text{det} \; A) (\text{det} \; B)\)
        \item \(\text{det} (2A) = 2^n \text{det} \; A\)
    \end{enumerate}
\end{theorem}

\begin{theorem}
    Row Operations and the Determinant:
    
    Let \(A\) be a square matrix.
    \begin{enumerate}
        \item If a multiple of a row of \(A\) is added to another row to produce \(B\), then \(\text{det} \; B = \text{det} \; A\).
        \item If two rows are interchanged to produce \(B\), then \(\text{det} \; B = -\text{det} \; A\).
        \item If one row of \(A\) is multiplied by a scaler \(k\) to produce \(B\), then \(\text{det} \; B = k \text{det} \; A\).
    \end{enumerate}
\end{theorem}

\noindent
\newline
Example: Compute \(\text{det} \; A=\begin{vmatrix}
    1 & -4 & 2 \\
    -2 & 8 & -9 \\
    -1 & 7 & 0
\end{vmatrix}\).

\begin{enumerate}
    \item \(2R_1 + R_2 \rightarrow R_2\)
    \item \(R_1 + R_3 \rightarrow R_3\)
    \item \(R_2 \leftrightarrow R_3\)
    \item \(-\frac{1}{5} R_3 \rightarrow R_3\)
\end{enumerate}

\begin{align}
    \begin{vmatrix}
        1 & -4 & 2 \\
        0 & 0 & -5 \\
        0 & 3 & 2
    \end{vmatrix} \\
    \begin{vmatrix}
        1 & -4 & 2 \\
        0 & 3 & 2 \\
        0 & 0 & -5
    \end{vmatrix} \\
    \begin{vmatrix}
        1 & -4 & 2 \\
        0 & 3 & 2 \\
        0 & 0 & 1
    \end{vmatrix}
\end{align}

\begin{align}
    E_4 E_3 E_2 E_1 A &= B \\
    -\frac{1}{5} * -1 * 1 * 1 * \text{det} \; A = \text{det} \; B &= 3 \\
    \text{det} \; A &= 15
\end{align}

\subsection{Volume from Determinants}
The absolute value of the determinant of an \(n \times n\) matrix represents the volume of a \(n\) dimensional parallepiped where the sides are the basis vectors of the matrix.

When a linear transformation occurs, the volume of the shape is scaled by the absolute value of the determinant.

\begin{theorem}
    If \(T_A : \Re^n \rightarrow \Re^n\), and \(S\) is some parallelogram in \(\Re^n\), then:
    \[\text{volume}(T_A(S)) = | \text{det} (A) | \times \text{volume}(S)\]
\end{theorem}
