\documentclass{article}

\usepackage{amsmath, amsthm, amssymb, amsfonts}
\usepackage{thmtools}
\usepackage{graphicx}
\usepackage{setspace}
\usepackage{geometry}
\usepackage{float}
\usepackage{hyperref}
\usepackage[utf8]{inputenc}
\usepackage[english]{babel}
\usepackage{framed}
\usepackage[dvipsnames]{xcolor}
\usepackage{tcolorbox}


\colorlet{LightGray}{White!90!Periwinkle}
\colorlet{LightOrange}{Orange!15}
\colorlet{LightGreen}{Green!15}
\colorlet{LightBlue}{Blue!15}

\newcommand{\HRule}[1]{\rule{\linewidth}{#1}}

\makeatletter
\renewcommand*\env@matrix[1][*\c@MaxMatrixCols c]{%
  \hskip -\arraycolsep
  \let\@ifnextchar\new@ifnextchar
  \array{#1}}
\makeatother

\newcommand{\spike}[2]% #1 = size of spike, #2 = centered text
{\bgroup
  \sbox0{#2}%
  \rlap{\usebox0}%
  \hspace{0.5\wd0}%
  \makebox[0pt][c]{\rule[\dimexpr \ht0+1pt]{0.5pt}{#1}}% top spike
  \makebox[0pt][c]{\rule[\dimexpr -\dp0-#1-1pt]{0.5pt}{#1}}% bottom spike
  \hspace{0.5\wd0}%
\egroup}

\declaretheoremstyle[name=Theorem,]{thmsty}
\declaretheorem[style=thmsty,numberwithin=section]{theorem}
\tcolorboxenvironment{theorem}{colback=LightGray}

\declaretheoremstyle[
    name=Definition,
    bodyfont=\upshape % This ensures definitions are upright (not italic)
]{defsty}
\declaretheorem[name=Definition,]{defsty}
\declaretheorem[style=defsty,numberlike=theorem]{definition}
\tcolorboxenvironment{definition}{colback=LightBlue}

\declaretheoremstyle[name=Proposition,]{prosty}
\declaretheorem[style=prosty,numberlike=theorem]{proposition}
\tcolorboxenvironment{proposition}{colback=LightOrange}

\declaretheoremstyle[name=Principle,]{prcpsty}
\declaretheorem[style=prcpsty,numberlike=theorem]{principle}
\tcolorboxenvironment{principle}{colback=LightGreen}

\setstretch{1.2}
\geometry{
    textheight=9in,
    textwidth=5.5in,
    top=1in,
    headheight=12pt,
    headsep=25pt,
    footskip=30pt
}

% ------------------------------------------------------------------------------

\begin{document}

% ------------------------------------------------------------------------------
% Cover Page and ToC
% ------------------------------------------------------------------------------

\title{ \normalsize \textsc{}
		\\ [2.0cm]
		\HRule{1.5pt} \\
		\LARGE \textbf{\uppercase{MATH 1553/1554\\ Linear Algebra}
		\HRule{2.0pt} \\ \vspace*{10\baselineskip}}
		}
\date{}
\author{\textbf{Saptak Das} \\ 
		Fall 2024}

\maketitle
\newpage

\tableofcontents
\newpage

% ------------------------------------------------------------------------------

\section{Overview}
\begin{itemize}
    \item Heavily conceptual.
    \begin{itemize}
        \item Careful choice of terms.
    \end{itemize}
    \item Steep Difficulty Change.
\end{itemize}

\noindent
\newline
\textbf{Linear algebra is everywhere:}
\begin{itemize}
    \item Your Computer
    \item Data Compression
    \item Quantum Physics
    \item Population Studies (Epidemiology)
    \item Google Page Rank and Singular Value Decomposition
\end{itemize}

\section{Systems of Linear Equations}
\begin{definition}
Multidimensional Vector Spaces
\begin{itemize}
    \item $\Re$: All Real Numbers
    \item $\Re^2$: All \((x, y)\) such that \(x\) and \(y\) are real.
    \item $\Re^3$: All \((x, y, z)\) such that \(x\), \(y\), and \(z\) are real.
    \item $\Re^n$: All \((x_1, x_2, \dots, x_n)\) such that \(x_1, x_2, \dots, x_n\) are real.
\end{itemize}
\end{definition}

\noindent
Consider \(S(x, y, 0)\), the x-y plane in \(\Re^3\).
\begin{itemize}
    \item \textbf{Question: Is S equal to \(\Re^3\)?}
    \item \textit{Answer: No. \(\Re^2\) has two coordinates. \(\Re^3\) has three coordinates.}
\end{itemize}

\subsection{Lines and Planes in \(\Re^n\)}
\textbf{Ex. 1:}
\(x + y = 1\) in \(\Re^2\)

\noindent
Parameterization:
\(x, 1-x\)

\noindent
\newline
\textbf{Ex. 2:}
\(3x - y + z = 3\) in \(\Re^3\)

\noindent
Parameterization:
\((x, y, 3 - 3x + y)\)

\noindent
\newline
\textbf{Ex. 3:}
\(x - 2y + z - 5w = 7\) in \(\Re^4\) aka "3-plane" in \(Re^4\)

\subsection{Linear Equations}
Examples of Linear Equations:
\begin{align}
    x +2y + 5z &= 3 \\
    x &= 5 - w \\
    2x_1 - 3 x_3 &= \pi + x_2 - ln(n)x_4
\end{align}

\noindent
Examples of Nonlinear Equations:
\begin{align}
    \sqrt{x} \\
    x^2 \\
    sin(x) \\
    ln(y) \\
    xy = 5
\end{align}

\subsubsection{Solution of Linear Equation}
The solution is the intersection of lines, planes, etc.
\begin{definition}
    A linear equation system is considered to be \textbf{consistent} if it has at least one solution. Otherwise, the system is considered to be \textbf{inconsistent}.
\end{definition}

\noindent
\newline
\textbf{Ex. 1: Consistent System. Solution is \((2, 0, 3)\).}
\begin{align}
    x - y + z &= 5 \\
    -x + y + 2z &= 4
\end{align}

\noindent
\newline
\textbf{Ex. 2: Inconsistent Linear Equations in (x, y)}
\begin{align}
    x + y &= 0 \\
    x + y &= 1 
\end{align}

\noindent
\newline
\textbf{Ex. 3: Consistent Linear Equations in (x, y). Solution is at \((3, 1).\)}
\begin{align}
    x - y &= 2 \\
    x + y &= 4
\end{align}

\noindent
\newline
\textbf{Ex. 4: Consistent Linear Equations in (x, y). Infinite number of solutions.}
\begin{align}
    x + y &= 0 \\
    3x + 3y &= 0
\end{align}

\subsection{Augmented Matrices}
\textbf{Ex. 1:}
\begin{align}
    x - y &= 10 \\
    3x + 6y &= 29 
\end{align}

\noindent
\newline
Augemented Matrix Form:
\begin{equation}
    \begin{pmatrix}[cc|c]
      1 & -1 & 10 \\
      3 & 6 & 29
    \end{pmatrix}
\end{equation}

\noindent
\newline
\textit{Solving Process using Row Reduction (Gaussian Elimination):}
\begin{enumerate}
    \item \(R_2 - 3R_1 \rightarrow R_2\)
    \item \(\frac{1}{9}R_2 \rightarrow R_2\)
\end{enumerate}

\begin{align}
    \begin{pmatrix}[cc|c]
      1 & -1 & 10 \\
      0 & 9 & -1
    \end{pmatrix} \\
    \begin{pmatrix}[cc|c]
      1 & -1 & 10 \\
      0 & 1 & -\frac{1}{9}
    \end{pmatrix}
\end{align}
\newline

\noindent
\newline
Back-substitution after Gaussian Elimination:
\begin{align}
    x-y &= 10 \\
    y = -\frac{1}{9}
\end{align}

\noindent
\(\therefore x=\frac{89}{9}\) and \(y=-\frac{1}{9}\)

\noindent
\newline
\textbf{OR:} Gauss-Jordan Elimination:
\begin{align}
    \begin{pmatrix}[cc|c]
      1 & 0 & \frac{89}{9} \\
      0 & 1 & -\frac{1}{9}
    \end{pmatrix}
\end{align}
\newline
\noindent
\(\therefore x=\frac{89}{9}\) and \(y=-\frac{1}{9}\)


\noindent
\textbf{Ex. 2:}
\begin{align}
    x + 2y + 3z &= 6 \\
    2x -3y + 2z &=14 \\
    3x + y -z &= -2
\end{align}

\noindent
\newline
Augemented Matrix Form:
\begin{equation}
    \begin{pmatrix}[ccc|c]
      1 & 2 & 3 & 6 \\
      2 & -3 & 2 & 14 \\
      3 & 1 & -1 & -2 
    \end{pmatrix}
\end{equation}

\noindent
\newline
\textit{Solving Process using Row Reduction:}
\begin{enumerate}
    \item \(R_2 - 2R_1 \rightarrow R_2\)
    \item \(R_3 - 3R_1 \rightarrow R_3\)
    \item \(R_2 \leftrightarrow R_3\)
    \item \(-\frac{1}{5}R_2 \rightarrow R_2\)
    \item \(R3 + 7R_2 \rightarrow R_3\)
    \item \(\frac{1}{10}R_3 \rightarrow R_3\)
\end{enumerate}

\begin{align}
    \begin{pmatrix}[ccc|c]
      1 & 2 & 3 & 6 \\
      0 & -7 & -4 & 2 \\
      3 & 1 & -1 & -2 
    \end{pmatrix} \\
    \begin{pmatrix}[ccc|c]
      1 & 2 & 3 & 6 \\
      0 & -7 & -4 & 2 \\
      0 & -5 & -10 & -20 
    \end{pmatrix} \\
    \begin{pmatrix}[ccc|c]
      1 & 2 & 3 & 6 \\
      0 & -5 & -10 & -20 \\
      0 & -7 & -4 & 2
    \end{pmatrix} \\
    \begin{pmatrix}[ccc|c]
      1 & 2 & 3 & 6 \\
      0 & 1 & 2 & 4 \\
      0 & -7 & -4 & 2
    \end{pmatrix} \\
    \begin{pmatrix}[ccc|c]
      1 & 2 & 3 & 6 \\
      0 & 1 & 2 & 4 \\
      0 & 0 & 10 & 30
    \end{pmatrix} \\
    \begin{pmatrix}[ccc|c]
      1 & 2 & 3 & 6 \\
      0 & 1 & 2 & 4 \\
      0 & 0 & 1 & 3
    \end{pmatrix}
\end{align}

\noindent
\newline
Back-substitution after Gaussian Elimination:
\begin{align}
    x + 2y + 3z &= 6 \\
    y + 2z &= 4 \\
    z &= 3
\end{align}

\noindent
\(\therefore x=1\), \(y=-2\), \(z = 3\)

\begin{enumerate}
    \item \(R_2 - 2R_3 \rightarrow R_2\)
    \item \(R_1 - 3R_3 \rightarrow R_1\)
    \item \(R_1 - 2R_2 \rightarrow R_1\)
\end{enumerate}

\noindent
\newline
\textbf{OR:} Gauss-Jordan Elimination:
\begin{align}
    \begin{pmatrix}[ccc|c]
      1 & 2 & 3 & 6 \\
      0 & 1 & 0 & -2 \\
      0 & 0 & 1 & 3
    \end{pmatrix} \\
    \begin{pmatrix}[ccc|c]
      1 & 2 & 0 & -3 \\
      0 & 1 & 0 & -2 \\
      0 & 0 & 1 & 3
    \end{pmatrix} \\
    \begin{pmatrix}[ccc|c]
      1 & 0 & 0 & 1 \\
      0 & 1 & 0 & -2 \\
      0 & 0 & 1 & 3
    \end{pmatrix}
\end{align}
\newline
\noindent
\(\therefore x=1\), \(y=-2\), \(z = 3\)

\begin{definition}
    Row-Echelon Form:
    \begin{enumerate}
        \item All zero rows (if any) are at the bottom.
        \item Each first non-zero entry (aka leading entry or pivot point) in a row is to the right of the first non-zero entry in the above row.
        \item Below any leading entry, all entries are zero.
    \end{enumerate}
    \[
        \begin{pmatrix}[ccc|c]
            a_1 & a_2 & a_3 & a_4 \\
            0 & a_5 & a_6 & a_7 \\
            0 & 0 & a_8 & a_9
        \end{pmatrix}
    \]
\end{definition}

\begin{definition}
    Reduced Row-Echelon:
    \begin{enumerate}
        \item All row-echelon form requirements must be satisfied.
        \item Each leading entry is 1.
        \item Each pivot point is the \textbf{only} non-zero entry in its column.
    \end{enumerate}
     \[
        \begin{pmatrix}[ccc|c]
            1 & 0 & 0 & a_1 \\
            0 & 1 & 0 & a_2 \\
            0 & 0 & 1 & a_3
        \end{pmatrix}
    \]
\end{definition}

\noindent
\textbf{Ex. 3:}
\begin{align}
    2x_1 - 3x_2 + x_3 + x_4 &= 6 \\
    x_1 - x_2 - x_4 &= 4 \\
    x_2 - x_3 &= 2
\end{align}

\noindent
\newline
Augemented Matrix Form:
\begin{equation}
    \begin{pmatrix}[cccc|c]
      2 & -3 & 1 & 1 & 6 \\
      1 & -1 & 0 & -1  & 4 \\
      0 & 1 & -1 & 0 & 2
    \end{pmatrix}
\end{equation}

\noindent
\newline
\textit{Solving Process using Row Reduction:}
\begin{enumerate}
    \item \(R_1 \leftrightarrow R_2\)
    \item \(R_2 - 2R_1 \rightarrow R_2\)
    \item \(-R_2 \rightarrow R_2\)
    \item \(R_3 - R_2 \rightarrow R_3\)
    \item \(\frac{1}{3} R_3 \rightarrow R_3\)
    \item \(3R_3 + R_2 \rightarrow R_2\)
    \item \(R_3 + R_1 \rightarrow R_1\)
    \item \(R_2 + R_1 \rightarrow R_1\)
\end{enumerate}

\begin{align}
    \begin{pmatrix}[cccc|c]
      1 & -1 & 0 & -1  & 4 \\
      2 & -3 & 1 & 1 & 6 \\
      0 & 1 & -1 & 0 & 2
    \end{pmatrix} \\
    \begin{pmatrix}[cccc|c]
      1 & -1 & 0 & -1  & 4 \\
      0 & -1 & 1 & 3 & -2 \\
      0 & 1 & -1 & 0 & 2
    \end{pmatrix} \\
    \begin{pmatrix}[cccc|c]
      1 & -1 & 0 & -1  & 4 \\
      0 & -1 & 1 & 3 & -2 \\
      0 & 1 & -1 & 0 & 2
    \end{pmatrix} \\
    \begin{pmatrix}[cccc|c]
      1 & -1 & 0 & -1  & 4 \\
      0 & 1 & -1 & -3 & 2 \\
      0 & 1 & -1 & 0 & 2
    \end{pmatrix} \\
    \begin{pmatrix}[cccc|c]
      1 & -1 & 0 & -1  & 4 \\
      0 & 1 & -1 & -3 & 2 \\
      0 & 0 & 0 & 3 & 0
    \end{pmatrix} \\
    \begin{pmatrix}[cccc|c]
      1 & -1 & 0 & -1  & 4 \\
      0 & 1 & -1 & -3 & 2 \\
      0 & 0 & 0 & 1 & 0
    \end{pmatrix} \\
    \begin{pmatrix}[cccc|c]
      1 & -1 & 0 & -1  & 4 \\
      0 & 1 & -1 & 0 & 2 \\
      0 & 0 & 0 & 1 & 0
    \end{pmatrix} \\
    \begin{pmatrix}[cccc|c]
      1 & -1 & 0 & 0  & 4 \\
      0 & 1 & -1 & 0 & 2 \\
      0 & 0 & 0 & 1 & 0
    \end{pmatrix} \\
\end{align}

\subsection{Parametric Forms}
\textbf{Write the solution set to \(x+y+z=10\).}
\begin{equation}
    \begin{pmatrix}[ccc|c]
        1 & 1 & 1 & 10
    \end{pmatrix}
\end{equation}

\noindent
\newline
The first one is a pivot. The other ones are free variables.

\begin{align}
    x &= 10 - y - z \\
    y &= y \\
    z &= z
\end{align}

\noindent
\newline
Geometrically, the system is a plane in \(\Re^3\).
\section{Matrix Equation}
\subsection{Linear Combinations and Span}
\begin{definition}
    Linear Combinations
    \newline
    A is a \(m\times n\) matrix with columns \(\Vec{a_1}, \dots, \Vec{a_n}\) and \(x\in \Re^n\), then the matrix-vector product \(A\Vec{x}\) is a linear combination of the columns of A:
    \begin{equation}
        A\Vec{x} = 
        \begin{pmatrix}
            \spike{15pt}{$\Vec{a_1}$}  & \spike{15pt}{$\Vec{a_2}$} & \spike{15pt}{$\Vec{a_3}$}
        \end{pmatrix}
        \begin{pmatrix}
            x_1 \\
            x_2 \\
            x_3
        \end{pmatrix} = x_1 \Vec{a_1} + x_2 \Vec{a_2} + x_3 \Vec{a_3}
    \end{equation}
\end{definition}

\begin{definition}
    Span 
    \newline
    The Span of a set of vectors \(\Vec{v_1}, \dots, \Vec{v_p}\) in \(Re^n\) is: \[\text{Span}\{\Vec{v_1}, \dots, \Vec{v_p}\} = \{x_1 \Vec{v_1} + \dots + x_2 \Vec{v_p} \; | \; x_1, \dots, x_p \in \Re\}\]
\end{definition}

\noindent
For what vectors \(\Vec{b} = \begin{pmatrix}
    b_1 \\ b_2 \\ b_3
\end{pmatrix}\), does the equation have a solution?

\begin{equation}
    \begin{pmatrix}
        1 & 3 & 4 \\
        2 & 8 & 4 \\
        0 & 1 & -2
    \end{pmatrix}
    \Vec{x} = \Vec{b}
\end{equation}

\noindent
\textbf{Solution: Row Reduce \([A | b]\)}
\begin{align}
    \begin{pmatrix}[ccc|c]
        1 & 3 & 4 & b_1 \\
        2 & 8 & 4 & b_2 \\
        0 & 1 & -2 & b_3
    \end{pmatrix} \\
    \begin{pmatrix}[ccc|c]
        1 & 3 & 4 & b_1 \\
        0 & 2 & -4 & -2b_1 + b_2 \\
        0 & 1 & -2 & b_3
    \end{pmatrix} \\
    \begin{pmatrix}[ccc|c]
        1 & 3 & 4 & b_1 \\
        0 & 1 & -2 & b_3 \\
        0 & 2 & -4 & -2b_1 + b_2
    \end{pmatrix} \\
    \begin{pmatrix}[ccc|c]
        1 & 3 & 4 & b_1 \\
        0 & 1 & -2 & b_3 \\
        0 & 0 & 0 & -2b_1 + b_2 - 2b_3
    \end{pmatrix}
\end{align}

\noindent
\(-2b_1 + b_2 - 2b_3 \neq 0\) is a pivot. Furthermore, the system will be consistent when \(-2b_1 + b_2 - 2b_3 = 0\).

\noindent
\newline
\textbf{Q: When is \(\Vec{b} = \begin{pmatrix}
    b_1 \\ b_2 \\ b_3
\end{pmatrix}\) in the span of \(\begin{pmatrix}
    1 \\ 2 \\ 0
\end{pmatrix}\), \(\begin{pmatrix}
    3 \\ 8 \\ 1
\end{pmatrix}\), \(\begin{pmatrix}
    4 \\ 4 \\ -2
\end{pmatrix}\)?}

\noindent
Same solution. \(\Vec{b}\) is in the span of the vectors when it can be expressed as a linear combination. In other words, some real scalars \(x_1, x_2, x_3\) can be used to create a linear combination of the three column vectors to result in \(\Vec{b}\).

\noindent
\newline
\textbf{Let A = \(\begin{pmatrix}
    1 & 3 & 4 \\
    -4 & 2 & -6 \\
    -3 & -2 & -7
    \end{pmatrix}\) and \(\Vec{b} = \begin{pmatrix}
        b_1 \\ b_2 \\ b_3
    \end{pmatrix}\). Is the equation \(A\Vec{x} = \Vec{b}\) consistent for all possible \(b_1, b_2, b_3\)?}

\noindent
Solution: Row Reduce A
\begin{align}
    \begin{pmatrix}
    1 & 3 & 4 \\
    -4 & 2 & -6 \\
    -3 & -2 & -7
    \end{pmatrix} \\
    \begin{pmatrix}
        1 & 3 & 4 \\
        0 & 7 & 5 \\
        0 & 0 & 0
    \end{pmatrix} \\
\end{align}

\noindent
If A does not have a pivot in every row, then some choice of \(\Vec{b}\) will make a pivot in \([A|b]\) inconsistent.

\subsection{Solution Sets of Linear Systems}
\begin{definition}
    Homogeneity
    Linear systems of the form \(A\Vec{x} = \Vec{0}\) are homogeneous.
    Linear systems of the form \(A\Vec{x} = \Vec{b}\) are inhomogeneous.
\end{definition}

\noindent
Because homogeneous systems always have the trivial solution, \(\Vec{x} = \Vec{0}\), the interesting question is how many solutions the system will have.

\noindent
\newline
\textbf{Solution Set for a Homogeneous System:}

\noindent
\textit{Row Reduce \([A|\Vec{0}]\)}

\begin{align}
    \begin{pmatrix}[ccc|c]
        1 & 3 & 1 & 0 \\
        2 & -1 & 5 & 0 \\
        1 & 0 & -2 & 0
    \end{pmatrix} \\
    \begin{pmatrix}[ccc|c]
        1 & 3 & 1 & 0 \\
        0 & -7 & -7 & 0 \\
        0 & -3 & -3 & 0
    \end{pmatrix} \\
    \begin{pmatrix}[ccc|c]
        1 & 3 & 1 & 0 \\
        0 & 1 & 1 & 0 \\
        0 & 1 & 1 & 0
    \end{pmatrix} \\
    \begin{pmatrix}[ccc|c]
        1 & 3 & 1 & 0 \\
        0 & 1 & 1 & 0 \\
        0 & 0 & 0 & 0
    \end{pmatrix} \\
    \begin{pmatrix}[ccc|c]
        1 & 0 & -2 & 0 \\
        0 & 1 & 1 & 0 \\
        0 & 0 & 0 & 0
    \end{pmatrix}
\end{align}

\noindent
\newline
Rewrite the Equations:
\begin{align}
    x_1 - 2x_3 &= 0 \\
    x_2 + x_3 &= 0 \\
    x_3 &= x_3
\end{align}

\noindent
\newline
Move free variables to RHS. This results in parametric equation form.
\begin{align}
    x_1 &= 2x_3 \\
    x_2 &= -x_3 \\
    x_3 &= x_3
\end{align}

\noindent
\newline
Therefore, the vector parametric form is:
\begin{equation}
    \Vec{x} = x_3 \begin{pmatrix}
        2 \\
        -1 \\
        1
    \end{pmatrix}
\end{equation}

\noindent
The solution is a line in \(\Re^3\).

\noindent
\newline
\textbf{Inhomogeneous Case:}
\begin{align}
    \begin{pmatrix}[ccc|c]
        1 & 3 & 1 & 9 \\
        2 & -1 & 5 & 11 \\
        1 & 0 & -2 & 6
    \end{pmatrix} \\
    \begin{pmatrix}[ccc|c]
        1 & 3 & 1 & 9 \\
        0 & -7 & -7 & -7 \\
        0 & -3 & -3 & -3
    \end{pmatrix} \\
    \begin{pmatrix}[ccc|c]
        1 & 3 & 1 & 9 \\
        0 & 1 & 1 & 1 \\
        0 & 1 & 1 & 1
    \end{pmatrix} \\
    \begin{pmatrix}[ccc|c]
        1 & 0 & -2 & 6 \\
        0 & 1 & 1 & 1 \\
        0 & 0 & 0 & 0
    \end{pmatrix}
\end{align}

\noindent
\newline
Rewrite the Equations:
\begin{align}
    x_1 - 2x_3 &= 6 \\
    x_2 + x_3 &= 1 \\
    x_3 &= x_3
\end{align}

\noindent
\newline
Move free variables to RHS. This results in parametric equation form.
\begin{align}
    x_1 &= 6 + 2x_3 \\
    x_2 &= 1 - x_3 \\
    x_3 &= x_3
\end{align}

\noindent
\newline
Therefore, the vector parametric form is:
\begin{equation}
    \Vec{x} = \begin{pmatrix}
        6 \\
        1 \\
        0
    \end{pmatrix} + x_3 \begin{pmatrix}
        2 \\
        -1 \\
        1
    \end{pmatrix}
\end{equation}

\noindent
The solution is a line in \(\Re^3\). The particular solution is \(\begin{pmatrix}
        6 \\
        1 \\
        0
\end{pmatrix}\). The general solutions to \(A\Vec{x}=\Vec{b}\) are general solutions of \(A\Vec{x}=\Vec{0}\) plus a particular solution.

\noindent
\newline
\textbf{Find general solutions to both \(A\Vec{x}=\Vec{b}\) and \(A\Vec{x}=\Vec{0}\).}
\begin{align}
    A&=\begin{pmatrix}
        1 & 3 & 0 & -4 \\
        2 & 6 & 0 & -8
    \end{pmatrix} \\
    \Vec{x} &= \begin{pmatrix}
        1 \\ 2 \\ 3 \\ 4
    \end{pmatrix} \\
    \Vec{b} &= \begin{pmatrix}
        -9 \\ -18
    \end{pmatrix}
\end{align}

\noindent
\newline
Row reduce \([A | b]\).
\begin{align}
    \begin{pmatrix}[cccc|c]
        1 & 3 & 0 & -4 & -9 \\
        2 & 6 & 0 & -8 & -18
    \end{pmatrix} \\
    \begin{pmatrix}[cccc|c]
        1 & 3 & 0 & -4 & -9 \\
        0 & 0 & 0 & 0 & 0
    \end{pmatrix}
\end{align}

\noindent
\newline
Rewrite the Equations:
\begin{align}
    x_1 + 3x_2 -4x_4 &= -9 \\
    x_2 &= x_2 \\
    x_3 &= x_3 \\
    x_4 &= x_4 \\
\end{align}

\noindent
\newline
Move free variables to RHS. This results in parametric equation form.
\begin{align}
    x_1 &= -9 - 3x_2 + 4 x_4 \\
    x_2 &= x_2 \\
    x_3 &= x_3 \\
    x_4 &= x_4 \\
\end{align}

\noindent
\newline
Therefore, the vector parametric form is:
\begin{equation}
    \Vec{x} = \begin{pmatrix}
        -9 \\
        0 \\
        0 \\
        0
    \end{pmatrix} + x_2 \begin{pmatrix}
        -3 \\
        1 \\
        0 \\
        0
    \end{pmatrix} + x_3 \begin{pmatrix}
        0 \\
        0 \\
        1 \\
        0
    \end{pmatrix} + x_4 \begin{pmatrix}
        4 \\
        0 \\
        0 \\
        1
    \end{pmatrix}
\end{equation}
The solution is a 3D-object in \(\Re^4\). The particular solution is \(\begin{pmatrix}
        -9 \\
        0 \\
        0 \\
        0
\end{pmatrix}\). The general solutions to \(A\Vec{x}=\Vec{b}\) are general solutions of \(A\Vec{x}=\Vec{0}\) plus a particular solution. Therefore, the solution to \(A\Vec{x} = \Vec{0}\) is:
\begin{equation}
    \Vec{x} = x_2 \begin{pmatrix}
        -3 \\
        1 \\
        0 \\
        0
    \end{pmatrix} + x_3 \begin{pmatrix}
        0 \\
        0 \\
        1 \\
        0
    \end{pmatrix} + x_4 \begin{pmatrix}
        4 \\
        0 \\
        0 \\
        1
    \end{pmatrix}
\end{equation}

\subsection{Linear Independence}
\begin{definition}
    An indexed set of vectors \(\{\Vec{v_1}, \dots, \Vec{v_p}\}\) in \(Re^n\) is said to be linearly independent if the vector equation:
    \[x_1\Vec{v_1} + x_2\Vec{v_2} + \dots + x_p \Vec{v_p} = 0\]
    has only the trivial solution. The set \(\{\Vec{v_1}, \dots, \Vec{v_p}\}\) is said to be linearly dependent if the vector equation \[x_1\Vec{v_1} + x_2\Vec{v_2} + \dots + x_p \Vec{v_p} = 0\] has an infinite number of solutions where some values in \(x_1, \dots, x_p\) are non-zero.
\end{definition}

\noindent
\newline
\textbf{Ex. 1: Is the following linearly independent: \(\begin{pmatrix}1 \\ 0 \\ 1\end{pmatrix}, \begin{pmatrix}2 \\ 1 \\ 3\end{pmatrix}, \begin{pmatrix}0 \\ 1 \\ 1\end{pmatrix}\)}

\noindent
\newline
Solution:
\begin{align}
    \begin{pmatrix}
        1 & 2 & 0 \\
        0 & 1 & 1 \\
        1 & 3 & 1
    \end{pmatrix} \\
    \begin{pmatrix}
        1 & 2 & 0 \\
        0 & 1 & 1 \\
        0 & 1 & 1
    \end{pmatrix} \\
    \begin{pmatrix}
        1 & 0 & -2 \\
        0 & 1 & 1 \\
        0 & 0 & 0
    \end{pmatrix} \\
\end{align}

\noindent
\newline
Rewrite the Equations:
\begin{align}
    x_1 - 2x_3 &= 0 \\
    x_2 + x_3 &= 0 \\
    x_3 &= x_3
\end{align}

\noindent
\newline
Move free variables to RHS. This results in parametric vector form.
\begin{equation}
    \Vec{x} = x_3 \begin{pmatrix}
        2 \\
        -1 \\
        1
    \end{pmatrix}
\end{equation}

\noindent
\newline
Therefore, the system is linearly dependent because there are non-trivial solutions to the coefficient matrix. Specifically, one dependence relation is:
\begin{equation}
    \begin{pmatrix}
        1 & 2 & 0 \\
        0 & 1 & 1 \\
        1 & 3 & 1
    \end{pmatrix} \begin{pmatrix}
        2 \\
        -1 \\
        1
    \end{pmatrix} = 2\begin{pmatrix}1 \\ 0 \\ 1\end{pmatrix} - \begin{pmatrix}2 \\ 1 \\ 3\end{pmatrix} + \begin{pmatrix}0 \\ 1 \\ 1\end{pmatrix} = \begin{pmatrix}0 \\ 0 \\ 0\end{pmatrix}
\end{equation}

\noindent
\newline
\textbf{Ex. 2: Is the following linearly independent: \(\begin{pmatrix}1 \\ 0 \\ 1\end{pmatrix}, \begin{pmatrix}2 \\ 2 \\ 0\end{pmatrix}, \begin{pmatrix}0 \\ 2 \\ 1\end{pmatrix}\)}

\noindent
\newline
Solution:
\begin{align}
    \begin{pmatrix}[ccc|c]
        1 & 2 & 0 & 0\\
        0 & 2 & 2 & 0 \\
        1 & 0 & 1 & 0
    \end{pmatrix} \\
    \begin{pmatrix}
        1 & 2 & 0 & 0\\
        0 & 1 & 1 & 0 \\
        0 & -2 & 1 & 0
    \end{pmatrix} \\
    \begin{pmatrix}
        1 & 0 & 0 & 0\\
        0 & 1 & 0 & 0 \\
        0 & 0 & 1 & 0
    \end{pmatrix} \\
\end{align}

\noindent
\newline
Rewrite the Equations:
\begin{align}
    x_1 &= 0 \\
    x_2 &= 0 \\
    x_3 &= 0
\end{align}

\noindent
\newline
Therefore, the system is linearly independent because there is only the trivial solution to the coefficient matrix.
\begin{equation}
    \begin{pmatrix}
        1 & 2 & 0 \\
        0 & 2 & 2 \\
        1 & 0 & 1
    \end{pmatrix} \begin{pmatrix}
        0 \\
        0 \\
        0
    \end{pmatrix} = 0\begin{pmatrix}1 \\ 0 \\ 1\end{pmatrix} + 0\begin{pmatrix}2 \\ 2 \\ 0\end{pmatrix} + 0\begin{pmatrix}0 \\ 2 \\ 1\end{pmatrix} = \begin{pmatrix}0 \\ 0 \\ 0\end{pmatrix}
\end{equation}

\noindent
\newline
\textbf{Ex. 3: Is the following linearly independent: \(\begin{pmatrix}1 \\ 0 \\ 0\end{pmatrix}, \begin{pmatrix}0 \\ 1 \\ 0\end{pmatrix}, \begin{pmatrix}3 \\ 2 \\ 0\end{pmatrix}\)}

\noindent
\newline
Solution: Linearly dependent. A dependence relation is:
\begin{equation}
    3\begin{pmatrix}1 \\ 0 \\ 0\end{pmatrix} + 2\begin{pmatrix}0 \\ 1 \\ 0\end{pmatrix} - \begin{pmatrix}3 \\ 2 \\ 0\end{pmatrix} = \begin{pmatrix}0 \\ 0 \\ 0\end{pmatrix}
\end{equation}
\section{Linear Transformations}
\subsection{Introduction}
Functions can be expressed as so:
\begin{equation}
    f: \; \Re \rightarrow \Re
\end{equation}

\noindent
A \(m \times n\) matrix transformation can be expressed as:
\begin{equation}
    T: \; \Re^n \rightarrow \Re^m, \; T(\vec{v}) = A \vec{v}
\end{equation}

\noindent
The vector \(T(\vec{v})\) is the image of \(\vec{v}\) under \(T\). The set of all possible images \(T(\vec{v})\) is the range.

\begin{definition}
    A function \(T\):  \(\Re^n\rightarrow\Re^m\) is linear if:
    \begin{itemize}
        \item \(T(\vec{u}+\vec{v}) = T(\vec{u}) + T(\vec{v})\) for all \(\vec{u}, \vec{v}\) in \(\Re^n\).
        \item \(T(c\vec{v}) = cT(\vec{v})\) for all \(\vec{v}\) in \(\Re^n\), and \(c\) in \(\Re\).
    \end{itemize}
\end{definition}

\begin{definition}
    The principle of superposition:
    
    If \(T\) is linear, then:
    \[T(c_1\Vec{v_1} + \dots + c_k \Vec{v_k}) = c_1 T(\vec{v_1}) + \dots + c_k T(\vec{v_k})\]
\end{definition}

\subsection{Standard Basis Vectors}
\noindent
For example, in \(\Re^3\),
\begin{align}
    \Vec{e_1} &= \begin{pmatrix}
        1 \\ 0 \\ 0
    \end{pmatrix} \\
    \Vec{e_2} &= \begin{pmatrix}
        0 \\ 1 \\ 0
    \end{pmatrix} \\
    \Vec{e_3} &= \begin{pmatrix}
        0 \\ 0 \\ 1
    \end{pmatrix}
\end{align}

\noindent
\newline
What is the linear transform \(T(\Vec{x})=A\Vec{x}\) to rotate counterclockwise by angle \(\theta\)?

\begin{align}
    A &= \begin{pmatrix}
        cos(\theta) & cos(\theta + \frac{\pi}{2}) \\
        sin(\theta) & sin(\theta + \frac{\pi}{2}) \\
    \end{pmatrix} \\
    &= \begin{pmatrix}
        cos(\theta) & -sin(\theta) \\
        sin(\theta) & cos(\theta) \\
    \end{pmatrix}
\end{align}

\noindent
\newline
Let \(T(\Vec{x}=A\Vec{x}\) be the transformation which first reflects vectors in \(\Re^2\) across the line \(y=0\) and then projects the resulting vector to the y-axis. Thus:
\begin{equation}
    A = \begin{pmatrix}
        0 & 0 \\
        0 & -1
    \end{pmatrix}
\end{equation}

\subsection{Onto and One-to-one Transformations}
\begin{definition}
    A linear transformation \(T: \; \Re^n \rightarrow \Re^m\) is onto if for all \(\Vec{b} \in \Re^m\) there is a \(\Vec{x} \in \Re^n\) so that \(T(\Vec{x})=\Vec{b}\). In other words, \(A\Vec{x}=\Vec{b}\) is always consistent. There are pivots in every row.
\end{definition}

\begin{definition}
    A linear transformation \(T: \; \Re^n \rightarrow \Re^m\) is one-to-one if for all \(\Vec{b} \in \Re^m\) there is at most one (possibly no) so that \(T(\Vec{x})=\Vec{b}\). There are pivots in every column.
\end{definition}

\noindent
\newline
\textbf{One-to-one but not onto?}

\noindent
Tall matrix. There is a pivot in every column, but not a pivot in every row.

\noindent
\newline
\textbf{Onto but not one-to-one?}

\noindent
Wide matrix. There is a pivot in every row, but not in every column.

\section{Matrix Operations}
\subsection{Properties}
\noindent
Given that A is an \(m \times n\) matrix.

\noindent
\newline
The following are \textbf{facts}:
\begin{enumerate}
    \item Associative: \((AB)C = A(BC)\)
    \item Left Distributive: \(A(B+C) = AB + AC\)
    \item Right Distributive: \((A+B)C = AC + BC\)
    \item Identity: \(I_m A = A I_n\)
\end{enumerate}

\noindent
\newline
The following are \textbf{non-facts}:
\begin{enumerate}
    \item Non-commutative: \(AB \ne BA\)
    \item Non-cancellation: \(AB = AC\), does not imply \(B=C\)
    \item Zero Divisors: \(AB = 0\) does not imply that either \(A=0\) or \(B=0\).
\end{enumerate}

\subsection{Transpose of a Matrix}
\noindent
\(A^T\) is the matrix whose columns are the rows of \(A\).

\noindent
\newline
Properties:

\begin{enumerate}
    \item \((A^T)^T=A\)
    \item \((A+B)^T = A^T + B^T\)
    \item \((rA)^T = rA^T\)
    \item \((AB)^T = B^T A^T\)
\end{enumerate}

\subsection{Matrix Powers}
\begin{equation}
    A^k = AA\dots A
\end{equation}

\noindent
\newline
When \(A\) is a diagonal matrix (non-zero terms only on the main diagonal), the power computation just exponentiates the main diagonal elements.
\section{Inverse of a Matrix}
\subsection{Definition}
\begin{definition}
    \(A \in R^{n\times n}\) is invertible if there is a \(C \in R^{n\times n}\) so that:
    \[AC = CA = I\]
\end{definition}

\subsection{Inverse of \(2\times 2 \) Matrix}
\begin{align}
    \begin{pmatrix}
        a & b \\
        c & d
    \end{pmatrix}^{-1} = \frac{1}{ad - bc}\begin{pmatrix}
        d & -b \\
        -c & a
    \end{pmatrix}
\end{align}

\subsection{Using Inverses to Solve Linear Systems}
\begin{align}
    3 x + 4 y &= 7 \\
    5 x + 6 y &= 7
\end{align}

\begin{align}
    \begin{pmatrix}
        3 & 4 \\
        5 & 6
    \end{pmatrix}^{-1} = \begin{pmatrix}
        -3 & 2 \\
        \frac{5}{2} & -\frac{3}{2}
    \end{pmatrix}
\end{align}

\begin{equation}
    \begin{pmatrix}
        -3 & 2 \\
        \frac{5}{2} & -\frac{3}{2}
    \end{pmatrix} \begin{pmatrix}
        7 \\ 7
    \end{pmatrix} = \begin{pmatrix}
        -7 \\ 7
    \end{pmatrix}
\end{equation}

\subsection{Properties of Inverses}
\(A\) and \(B\) are invertible \(n\times n\) matrices.
\begin{enumerate}
    \item \((A^{-1})^{-1}=A\)
    \item \((AB)^{-1}=B^{-1}A^{-1}\)
    \item \((A^T)^{-1} = (A^{-1})^T\)
\end{enumerate}

\subsection{General Algorithm for Computing \(A^{-1}\)}
\begin{align}
    \begin{pmatrix}[ccc|ccc]
        0 & 1 & 2 & 1 & 0 & 0 \\
        1 & 0 & 3 & 0 & 1 & 0 \\
        0 & 0 & 1 & 0 & 0 & 1
    \end{pmatrix} \\
    \begin{pmatrix}[ccc|ccc]
        1 & 0 & 3 & 0 & 1 & 0 \\
        0 & 1 & 2 & 1 & 0 & 0 \\
        0 & 0 & 1 & 0 & 0 & 1
    \end{pmatrix} \\
\end{align}

\noindent
\newline
Therefore, \(A^{-1}\) is \(\begin{pmatrix}
    0 & 1 & 0 \\
    1 & 0 & 0 \\
    0 & 0 & 1
\end{pmatrix}\). Thus, the inverse of a matrix is the product of the elementary matrices that turned the \(A\) matrix to RREF.

\subsection{Invertible Matrix Theorem}
Criteria for having an inverse are below. All statements are equivalent. Thus, if one statement is true, all are true. If one statement is false, all are false. \(A\) is an \(n \times n\) square matrix.

\begin{enumerate}
    \item \(A\) is invertible.
    \item \(A\) is row equivalent to \(I_n\).
    \item \(A\) has \(n\) pivotal columns.
    \item \(A \vec{x} = \Vec{0}\) only has the trivial solution.
    \item The columns of A are linearly independent.
    \item The linear transformation \(\Vec{x} \rightarrow A \Vec{x}\) is one-to-one.
    \item The equation \(A \Vec{x} = \Vec{b}\) has a solution for all \(\Vec{b}\in \Re^n\).
    \item The columns of \(A\) span \(\Re^n\).
    \item The linear transformation \(\Vec{x} \rightarrow A \Vec{x}\) is onto.
    \item There is a \(n \times n\) matrix \(C\) so that \(C A = I_n\). (Left Inverse).
    \item There is a \(n \times n\) matrix \(D\) so that \(A D = I_n\). (Right Inverse).
    \item \(A^T\) is invertible.
    \item The columns of \(A\) are a basis for \(\Re^n\).
    \item \(\text{Col} \; A = \Re^n\)
    \item \(\text{rank} \; A = \text{dim}(\text{Col} \; A) = n\)
    \item \(\text{Null} \; A = \{\Vec{0}\}\)
    \item \((\text{Col} \; A)^\perp = \{\Vec{0}\}\)
    \item \((\text{Nul} \; A)^\perp = \Re^n\)
    \item \(\text{Row} \; A = \Re^n\)
    \item \(A\) has \(n\) nonzero singular values.
\end{enumerate}

\begin{theorem}
    If \(A\) and \(B\) are \(n \times n\) matrices and \(A B = I\), then \(A = B^{-1}\) and \(B = A^{-1}\).
\end{theorem}

\noindent
\newline
\textbf{Ex: Is this matrix invertible?}
\begin{align}
    \begin{pmatrix}
        1 & 0 & -2 \\
        3 & 1 & -2 \\
        -5 & -1 & 9
    \end{pmatrix} \\
    \begin{pmatrix}
        1 & 0 & -2 \\
        0 & 1 & 4 \\
        0 & 0 & 3
    \end{pmatrix}
\end{align}

\noindent
Since the matrix has pivots in every column, the matrix is invertible.

\begin{definition}
    A singular matrix is non-invertible. A non-singular matrix is invertible.
\end{definition}

\subsection{Partitioned Matrix}
The matrix \(A\) can be represented as \(\begin{pmatrix}
    A_{1,1} & A _{1,2} \\
    A_{2,1} & A_{2,2}
\end{pmatrix}\).

\noindent
\newline
\textbf{Ex: Find the inverse of \(\begin{pmatrix}
    A & B \\
    0 & C
\end{pmatrix}\).}

\begin{equation}
    \begin{pmatrix}
        A & B \\
        0 & C
    \end{pmatrix}
    \begin{pmatrix}
        X & Y \\
        W & Z
    \end{pmatrix} =
    \begin{pmatrix}
        AX + BW & AY + BZ \\
        0X + CW & 0Y + CZ
    \end{pmatrix} =
    \begin{pmatrix}
        I_n & 0 \\
        0 & I_n
    \end{pmatrix}
\end{equation}

\noindent
Therefore, \(Z = C^{-1}\), \(W = 0\), \(X = A^{-1}\), \(Y=-A^{-1}BC^{-1}\).

\begin{equation}
    \begin{pmatrix}
        X & Y \\
        W & Z
    \end{pmatrix} = \begin{pmatrix}
        A^{-1} & -A^{-1}BC^{-1} \\
        0 & C^{-1}
    \end{pmatrix}
\end{equation}

\subsection{Strassen Algorithm}
Matrix multiplication is \(O(n^3)\). Strassen's Algorithm uses matrix partitions to get to \(O(n^{2.803\dots})\).
\section{Matrix Factorization}
Examples:
\begin{align}
    A &= LU \\
    A &= QR \\
    A &= PDP^{-1} \\
    A &= U \Sigma V^T
\end{align}

\subsection{LU Decomposition}
\begin{theorem}
    If \(A\) is an \(m \times n\) matrix that can be row reduced to echelon form without row exchanges, then \(A = LU\). \(L\) is a lower triangular \(m \times m\) matrix with 1's on the diagonal. \(U\) is an echelon form of \(A\).
\end{theorem}

\begin{definition} Types of Matrices
    \newline
    Upper Triangular Matrix: if \(a_{i,j} = 0\) for \(i > j\).
    \newline
    Lower Triangular Matrix: if \(a_{i,j} = 0\) for \(i < j\).
    \newline
    Diagonal: Both upper and lower, if \(a_{i,j} = 0\) for \(i \ne j\).
\end{definition}

\noindent
\newline
Example 1:
\begin{enumerate}
    \item \(4R_1 + R_2 \rightarrow R_2\)
    \item \(-2R_1 + R_3 \rightarrow R_3\)
    \item \(-5R_1 + R_3 \rightarrow R_3\)
\end{enumerate}

\begin{align}
    A &= \begin{pmatrix}
        4 & -3 & -1 & 5 \\
        -16 & 12 & 2 & -17 \\
        8 & -6 & -12 & 22
    \end{pmatrix} \\
    &\begin{pmatrix}
        4 & -3 & -1 & 5 \\
        0 & 0 & -2 & 3 \\
        0 & 0 & -10 & 12
    \end{pmatrix} \\
    U &= \begin{pmatrix}
        4 & -3 & -1 & 5 \\
        0 & 0 & -2 & 3 \\
        0 & 0 & 0 & -3
    \end{pmatrix}
\end{align}

\noindent
Use the scaling factors to determine \(L\),
\begin{align}
    L = \begin{pmatrix}
        1 & 0 & 0 \\
        -4  & 1 & 0 \\
        2 & 5 & 1
    \end{pmatrix}
\end{align}

\subsection{Why LU Decomposition works?}
\begin{align}
    E_3 E_2 E_1 A &= U \\
    A &= E_1^{-1} E_2^{-1} E_3^{-1} U \\
    A &= L U
\end{align}

\noindent
\newline
LU decompositions do not always exist and the decompositions are not always unique.

\subsection{Solving using LU Decomposition}
\begin{enumerate}
    \item \(L\Vec{y}=\Vec{b}\)
    \item \(U\Vec{x}=\Vec{y}\)
\end{enumerate}

\noindent
\newline
Example:
\begin{align}
    A &= LU = \begin{pmatrix}
        1 & 0 & 0 & 0 \\
        1 & 1 & 0 & 0 \\
        1 & 2 & 1 & 0 \\
        0 & -1 & -1 & 1
    \end{pmatrix}\begin{pmatrix}
        1 & 1 & 0 \\
        0 & -1 & -1 \\
        0 & 0 & 2 \\
        0 & 0 & 0
    \end{pmatrix} \\
    \Vec{b} &= \begin{pmatrix}
        16 \\ 2 \\ -4 \\6
    \end{pmatrix}
\end{align}

\noindent
Step 1: Row reduce \((L|\Vec{b})\).
\begin{equation}
    \begin{pmatrix}[cccc|c]
        1 & 0 & 0 & 0 & 16 \\
        0 & 1 & 0 & 0 & -14 \\
        0 & 0 & 1 & 0 & 8 \\
        0 & 0 & 0 & 1 & 0
    \end{pmatrix}
\end{equation}

\noindent
Step 2: Row reduce \((U|\Vec{y})\).
\begin{equation}
    \begin{pmatrix}[ccc|c]
        1 & 0 & 0 & 6 \\
        0 & 1 & 0 & 10 \\
        0 & 0 & 1 & 4 \\
        0 & 0 & 0 & 0
    \end{pmatrix}
\end{equation}

\noindent
\newline
Therefore, \(\Vec{x} = \begin{pmatrix}
    6 \\ 10 \\ 4
\end{pmatrix}\).
\section{Vector Subspaces}
\subsection{Definitions}
\begin{definition}
    A subset \(H\) of \(\Re^n\) is a subspace if it is closed under scalar multiples and vector addition. That is: for any \(c \in \Re\) and for \(\Vec{u},\Vec{v}\in H\).
    \begin{enumerate}
        \item \(c\Vec{u} \in H\)
        \item \(\Vec{u} + \Vec{v} \in H\)
    \end{enumerate}
\end{definition}

\begin{definition}
    Given an \(m \times n\) matrix \(A = [\Vec{a_1} \dots \Vec{a_n}]\)
    \begin{enumerate}
        \item The column space of \(A\), \(\text{Col} \; A\), is the subspace of \(\Re^m\) spanned by \(\Vec{a_1}, \dots, \Vec{a_n}\).
        \item The null space of \(A\), \(\text{Null} \; A\), is the subspace of \(\Re^n\) spanned by the set of all vectors \(\Vec{x}\) that solve \(A \Vec{x} = \Vec{0}\)
    \end{enumerate}
\end{definition}
\section{Dimension and Rank}
\begin{definition}
    The dimension of a subspace \(H\) of \(\Re^n\) is the number of basis vectors in any basis of \(H\). We define \(\text{dim}\{0\}=0\).
\end{definition}

\subsection{Choice of Basis}
The standard basis of \(\Re^3\) is \(B_1= \left\{ \begin{pmatrix}
    1 \\ 0 \\ 0
\end{pmatrix}, \begin{pmatrix}
    0 \\ 1 \\ 0
\end{pmatrix}, \begin{pmatrix}
    0 \\ 0 \\ 1
\end{pmatrix}
\right\}\).

\noindent
\newline
However, another possible choice of basis in \(\Re^3\) is \(B_2= \left\{ \begin{pmatrix}
    1 \\ 0 \\ 0
\end{pmatrix}, \begin{pmatrix}
    1 \\ 1 \\ 0
\end{pmatrix}, \begin{pmatrix}
    1 \\ 1 \\ 1
\end{pmatrix}
\right\}\).

\subsection{Dimension of Null and Column Spaces}
\(\text{dim}(\text{Null} \; A)\) is the number of free variables. \(\text{dim}(\text{Col} \; A)\) is the number of pivots.
\section{Determinants}
\subsection{Definition}
\begin{definition}
    Definition of Determinant
    Suppose \(A\) is \(n \times n\) and has elements \(a_{ij}\).
\begin{enumerate}
    \item If \(n=1\), \(A=a_{11}\), and has determinant \(\text{det} \; A = a_{11}\).
    \item Inductive case: for \(n > 1\),
    \[\text{det} \; A = a_{11} \text{det} \; A_{11} - a_{12} \text{det} \; A_{12} + \dots + (-1)^{1+n} a_{1n} \text{det} \; A_{1n}\]
    where \(A_{ij}\) is the submatrix obtained by eliminating row \(i\) and column \(j\) of \(A\).
\end{enumerate}
\end{definition}

\begin{definition}
    Cofactors

    The \((i, j)\) cofactor of an \(n\times n\) matrix A is:
    \[C_{ij} = (-1)^{i+j} \text{det} \; A_{ij}\]
\end{definition}

\begin{theorem}
    The determinant of a matrix \(A\) can be computed down any row or column of the matrix. For instance, down the jth column, the determinant is:
    \[\text{det} \; A = a_{1j} C_{1j} + a_{2j} C_{2j} + \dots + a_{nj} C_{nj}\]
\end{theorem}

\subsection{Determinant of a \(3 \times 3\) Matrix}
\begin{align}
    \text{det} \begin{pmatrix}
        1 & -5 & 0 \\
        2 & 4 & -1 \\
        0 & 2 & 0
    \end{pmatrix} = 1 \begin{vmatrix}
        4 & -1 \\ 2 & 0
    \end{vmatrix} - (-5) \begin{vmatrix}
        2 & -1 \\ 0 & 0
    \end{vmatrix} + 0 \begin{vmatrix}
        2 & 4 \\ 0 & 2
    \end{vmatrix} = 2
\end{align}

\subsection{Determinants of Triangular Matrices}
\begin{theorem}
    If \(A\) is a triangular matrix then,
    \[\text{det} \; A = a_{11} a_{22} a_{33} \dots a_{nn}\]
\end{theorem}

\subsection{Properties of the Determinant}
\begin{theorem}
    For any square matrices \(A\) and \(B\), we can show the following.
    \begin{enumerate}
        \item \(\text{det} \; A = \text{det} \; A^T\)
        \item \(A\) is invertible iff \(\text{det} \; A \ne 0\).
        \item \(\text{det} (A B)= (\text{det} \; A) (\text{det} \; B)\)
        \item \(\text{det} (2A) = 2^n \text{det} \; A\)
    \end{enumerate}
\end{theorem}

\begin{theorem}
    Row Operations and the Determinant:
    
    Let \(A\) be a square matrix.
    \begin{enumerate}
        \item If a multiple of a row of \(A\) is added to another row to produce \(B\), then \(\text{det} \; B = \text{det} \; A\).
        \item If two rows are interchanged to produce \(B\), then \(\text{det} \; B = -\text{det} \; A\).
        \item If one row of \(A\) is multiplied by a scaler \(k\) to produce \(B\), then \(\text{det} \; B = k \text{det} \; A\).
    \end{enumerate}
\end{theorem}

\noindent
\newline
Example: Compute \(\text{det} \; A=\begin{vmatrix}
    1 & -4 & 2 \\
    -2 & 8 & -9 \\
    -1 & 7 & 0
\end{vmatrix}\).

\begin{enumerate}
    \item \(2R_1 + R_2 \rightarrow R_2\)
    \item \(R_1 + R_3 \rightarrow R_3\)
    \item \(R_2 \leftrightarrow R_3\)
    \item \(-\frac{1}{5} R_3 \rightarrow R_3\)
\end{enumerate}

\begin{align}
    \begin{vmatrix}
        1 & -4 & 2 \\
        0 & 0 & -5 \\
        0 & 3 & 2
    \end{vmatrix} \\
    \begin{vmatrix}
        1 & -4 & 2 \\
        0 & 3 & 2 \\
        0 & 0 & -5
    \end{vmatrix} \\
    \begin{vmatrix}
        1 & -4 & 2 \\
        0 & 3 & 2 \\
        0 & 0 & 1
    \end{vmatrix}
\end{align}

\begin{align}
    E_4 E_3 E_2 E_1 A &= B \\
    -\frac{1}{5} * -1 * 1 * 1 * \text{det} \; A = \text{det} \; B &= 3 \\
    \text{det} \; A &= 15
\end{align}

\subsection{Volume from Determinants}
The absolute value of the determinant of an \(n \times n\) matrix represents the volume of a \(n\) dimensional parallepiped where the sides are the basis vectors of the matrix.

When a linear transformation occurs, the volume of the shape is scaled by the absolute value of the determinant.

\begin{theorem}
    If \(T_A : \Re^n \rightarrow \Re^n\), and \(S\) is some parallelogram in \(\Re^n\), then:
    \[\text{volume}(T_A(S)) = | \text{det} (A) | \times \text{volume}(S)\]
\end{theorem}

\section{Markov Chains}
\subsection{Introduction}
\begin{enumerate}
    \item A small town has two libraries, A and B.
    \item After 1 month, among the books checked out of A,
        \begin{enumerate}
            \item 80\% returned to A.
            \item 20\% returned to B.
        \end{enumerate}
    \item After 1 month, among the books checked out of B,
        \begin{enumerate}
            \item 30\% returned to A.
            \item 70\% returned to B.
        \end{enumerate}    
\end{enumerate}

\noindent
State after \(n\) months is:

\begin{equation}
    X_n = 1000 x_n = 1000 \times \begin{pmatrix}
        0.8 & 0.3 \\ 0.2 & 0.7 
    \end{pmatrix}^n \begin{pmatrix}
        0.5 \\
        0.5
    \end{pmatrix}
\end{equation}

\noindent
Where:
\begin{itemize}
    \item \(x_n\) is the nth element of the Markov Chain.
    \item \(X_n\) is the nth state of the system.
\end{itemize}

\subsection{Steady State of a Markov Chain}
\begin{definition} Markov Chain Definitions:
    \begin{enumerate}
        \item A probability vector is a vector, \(\Vec{x}\), with non-negative elements that sum to 1.
        \item A stochastic matrix is a square matrix, \(P\), whose columns are probability vectors.
        \item A Markov Chain is a sequence of probability vectors \(\Vec{x_k}\), and a stochastic matrix \(P\), such that:
        \[\Vec{x_{k+1}} = P \Vec{x_k},\; k=0,1,2,\dots\]
        \item A steady-state vector for P is a vector \(\Vec{q}\) such that \(P\Vec{q}=\Vec{q}\).
    \end{enumerate}
\end{definition}


\noindent
Example 1: Find a steady-state vector for the stochastic matrix.
\begin{equation}
    \begin{pmatrix}
        0.8 & 0.3 \\
        0.2 & 0.7
    \end{pmatrix}
\end{equation}

\noindent
Since:
\begin{align}
    P \Vec{x} &= \Vec{x} \\
    P \Vec{x} - \Vec{x} &= \Vec{0} \\
    P \Vec{x} - I \Vec{x} &= \Vec{0} \\
    (P - I) \Vec{x} &= \Vec{0}
\end{align}

\begin{align}
    \text{rref}(P - I) = \begin{pmatrix}
        1 & - \frac{3}{2} \\
        0 & 0
    \end{pmatrix}
\end{align}

\noindent
Determine \(x_2\) by ensuring that \(x_1 + x_2 = 1\).
\begin{align}
    x_2 = \frac{1}{\frac{3}{2}+1} = \frac{2}{5}
\end{align}

\noindent
Thus, \(\Vec{q} = \begin{pmatrix}
    \frac{3}{5} & \frac{2}{5}
\end{pmatrix}\).

\subsection{Convergence for a Regular Stochastic Matrix}
\begin{definition}
    A stochastic matrix \(P\) is regular if there is some k such that \(P^k\) only contains strictly positive entries.
\end{definition}
\begin{theorem}
    If \(P\) is a regular stochastic matrix, then \(P\) has a unique steady-state vector \(\Vec{q}\), and \(\Vec{x_k+1}=P\Vec{x_k}\) converges to \(\Vec{q}\) as \(k \rightarrow \infty\).
\end{theorem}

\noindent
Example 2: Find a steady-state vector for the stochastic matrix.
\begin{equation}
    \begin{pmatrix}
        0.8 & 0.1 & 0.2 \\
        0.2 & 0.6 & 0.3 \\
        0.0 & 0.3 & 0.5
    \end{pmatrix}
\end{equation}

\noindent
Since \(P\) is regular, a unique steady-state vector must exist.

\begin{equation}
    P-I = \begin{pmatrix}
        -0.2 & 0.1 & 0.2 \\
        0.2 & -0.4 & 0.3 \\
        0.0 & 0.3 & -0.5
    \end{pmatrix}
\end{equation}

\begin{align}
    \text{rref}(P-I) &= \begin{pmatrix}
        1 & 0 & -\frac{11}{6} \\
        0 & 1 & -\frac{5}{3} \\
        0 & 0 & 0
    \end{pmatrix} \\
    \Vec{q} &= \begin{pmatrix}
        \frac{11}{27} \\
        \frac{10}{27} \\
        \frac{6}{27}
    \end{pmatrix}
\end{align}
\section{Eigenvectors and Eigenvalues}
\subsection{Introduction}
\begin{definition}
    If \(A \in \Re^{n\times n}\), and there is a \(\Vec{v} \ne \Vec{0}\) in \(\Re^n\), and
    \[A\Vec{v} = \lambda \Vec{v}\]
    then \(\Vec{v}\) is an eigenvector for \(A\), and \(\lambda \in \mathbb{C}\) is the corresponding eigenvalue.
\end{definition}

\noindent
Example 1: For \(A = \begin{pmatrix}
    1 & 1 \\ 1 & 1
\end{pmatrix}\), are the following eigenvectors?
\begin{enumerate}
    \item \(\begin{pmatrix}
        1 \\ 1 
    \end{pmatrix}\): Yes. \(\lambda=2\).
    \item \(\begin{pmatrix}
        1 \\ -1
    \end{pmatrix}\): Yes. \(\lambda=0\).
    \item \(\begin{pmatrix}
        0 \\0 
    \end{pmatrix}\): No. The \(\Vec{0}\) cannot be an eigenvector.
\end{enumerate}

\noindent
Example 2: Confirm that \(\lambda=3\) is an eigenvalue of \(A = \begin{pmatrix}
    2 & -4 \\ -1 & -1
\end{pmatrix}\).

\begin{align}
    A\Vec{v} &= \lambda \Vec{v} \\
    A\Vec{v} - \lambda I \Vec{v} &= \Vec{0} \\
    (A-\lambda I) \Vec{v} &= \Vec{0}
\end{align}

\noindent
In this case, \(\lambda = 3\).

\begin{equation}
    A - 3I = \begin{pmatrix}
        -1 & -4 \\ -1 & -4
    \end{pmatrix} \rightarrow \begin{pmatrix}
        1 & 4 \\ 0 & 0
    \end{pmatrix}
\end{equation}

\noindent
Thus, \(\begin{pmatrix}
    -4 \\ 1
\end{pmatrix}\) is the eigenvector when \(\lambda = 3\).

\subsection{Eigenspace}
\begin{definition}
    Suppose \(A \in \Re^{n\times n}\). The eigenvectors for a given \(\lambda\) span a subspace of \(\Re^n\) called the \(\lambda\)-eigenspace of A. The \(\lambda\)-eigenspace of a matrix \(A\) is \(\text{Nul}(A-\lambda I)\).
\end{definition}

\noindent
Example 1: let \(A= \begin{pmatrix}
    5 & -6 \\ 3 & -4
\end{pmatrix}\) and \(\lambda=2\).

\begin{equation}
    A - 2 I = \begin{pmatrix}
        3 & -6 \\ 3 & -6
    \end{pmatrix} \rightarrow \begin{pmatrix}
        1 & -2 \\ 0 & 0
    \end{pmatrix}
\end{equation}

\noindent
Thus, a basis for the eigenspace is \(\begin{pmatrix}
    2 \\ 1
\end{pmatrix}\).

\noindent
Example 2: let \(A= \begin{pmatrix}
    4 & -1 & 6 \\
    2 & 1 & 6 \\
    2 & -1 & 8
\end{pmatrix}\) and \(\lambda=2\).

\begin{equation}
    A - 2 I = \begin{pmatrix}
    2 & -1 & 6 \\
    2 & -1 & 6 \\
    2 & -1 & 6
\end{pmatrix} \rightarrow \begin{pmatrix}
    1 & -\frac{1}{2} & 3 \\
    0 & 0 & 0 \\
    0 & 0 & 0
\end{pmatrix}
\end{equation}

\begin{equation}
    \Vec{x} = x_1 \begin{pmatrix}
        \frac{1}{2}  \\ 1 \\ 0 
    \end{pmatrix} + x_2 \begin{pmatrix}
        -3 \\ 0 \\ 1
    \end{pmatrix}
\end{equation}

\noindent
Therefore, the basis for the eigenspace is \(\left\{\begin{pmatrix}
        \frac{1}{2}  \\ 1 \\ 0 
    \end{pmatrix},\begin{pmatrix}
        -3 \\ 0 \\ 1
    \end{pmatrix}\right\}\).

\subsection{IMT Theorems}
\begin{theorem} IMT (Continued.)
    \begin{enumerate}
        \item The diagonal elements of a triangular matrix are its eigenvalues.
        \item \(A\) is invertible \(\Leftrightarrow 0\) is not an eigenvalue of \(A\).
        \item If \(\Vec{v_1}, \Vec{v_2}, \dots, \Vec{v_k}\) are eigenvectors that correspond to distinct eigenvalues, then \(\Vec{v_1}, \Vec{v_2}, \dots, \Vec{v_k}\) are linearly independent.
    \end{enumerate}
\end{theorem}

\subsection{Characteristic Polynomial}
The quantity \(\text{det} (A -\lambda I)\) is the characteristic polynomial of \(A\). The quantity \(\text{det} (A -\lambda I)=0\) is the characteristic equation of \(A\). The roots of the characteristic polynomial are the eigenvalues of A.

\noindent
\newline
Example 1:
\begin{align}
    p(\lambda) &= \text{det} \begin{pmatrix}
        5-\lambda & 2 \\ 2 & 1-\lambda
    \end{pmatrix} = \lambda^2 -6 \lambda + 1 \\
    \lambda &= \frac{6 \pm \sqrt{32}}{2} = 3 \pm 2\sqrt{2}
\end{align}

\noindent
\newline
General Case:
\begin{align}
    \text{det}\begin{pmatrix}
        a-\lambda & b \\ c & d-\lambda
    \end{pmatrix} = \lambda^2 - (a+d)\lambda + ad - bc
\end{align}

\noindent
Therefore, \(p(\lambda) = \lambda^2 - \text{tr} \; A + \text{det} \; A\).

\subsection{Multiplicities}
\begin{definition} Algebraic Multiplicity

    The algebraic multiplicity of an eigenvalue is its multiplicity as a root of the characteristic polynomial.
\end{definition}

\begin{definition} Geometric Multiplicity

    The geometric multiplicity of an eigenvalue \(\lambda\) is the dimension of \(\text{Null} (A-\lambda I)\).
    
\end{definition}

\noindent
\newline
Example of a \(4\times 4\) matrix with \(\lambda = 3\) as the only eigenvalue, but the geometric multiplicity of \(\lambda = 3\) is one.

\begin{equation}
    A = \begin{pmatrix}
        3 & 1 & 2 & 2 \\ 0 & 3 & 1 & 2 \\ 0 & 0 & 3 & 1 \\ 0 & 0 & 0 & 3
    \end{pmatrix}
\end{equation}

\subsection{Similar Matrices}
\begin{definition}
    Two \(n \times n\) matrices A and B are similar if there is a matrix P so that \(A = PBP^{-1}\).
\end{definition}

\begin{theorem}
    If \(A\) and \(B\) are similar, then they have the same characteristic polynomial.
\end{theorem}

\subsection{Applications of Eigenvalues and Eigenvectors}
\noindent
Consider the Markov Chain:
\begin{equation}
    \Vec{x_{k+1}} = P \Vec{x_0} = \begin{pmatrix}
        0.6 & 0.4 \\ 0.4 & 0.6
    \end{pmatrix}
\end{equation}

\section{Diagonalizability}
\subsection{Powers of Diagonal Matrices}
\begin{align}
    A &= \begin{pmatrix}
        3 & 0 \\ 0 & 0.5
    \end{pmatrix} \\
    A^2 &= \begin{pmatrix}
        3^2 & 0 \\ 0 & 0.5^2
    \end{pmatrix} \\
    A^k &= \begin{pmatrix}
        3^k & 0 \\ 0 & 0.5^k
    \end{pmatrix}
\end{align}

\subsection{Diagonalization}
\begin{definition}
    Suppose \(A \in \Re^{n \times n}\). We say that \(A\) is diagonalizable if it is similar to a diagonal matrix, \(D\). The columns of \(P\) are the eigenvectors of \(A\). \(D\) is a diagonal matrix with the corresponding eigenvalues along the main diagonal. That is, we can write:
    \[A = P D P^{-1}\]
\end{definition}

\noindent
Suppose:
\begin{align}
    D &= \begin{pmatrix}
        \lambda_1 & 0 & 0 \\
        0 & \lambda_2 & 0 \\
        0 & 0 & \lambda_3
    \end{pmatrix} \\
    P &= \begin{pmatrix}
            \spike{15pt}{$\Vec{v_1}$}  & \spike{15pt}{$\Vec{v_2}$} & \spike{15pt}{$\Vec{v_3}$}
        \end{pmatrix}
\end{align}

\noindent
\newline
\textbf{Example 1: Diagonalize if possible.}
\begin{equation}
    \begin{pmatrix}
        2 & 6 \\ 0 & -1
    \end{pmatrix}
\end{equation}

\noindent
\(\lambda_1 = 2\), \(\lambda_2 = -1\) because upper triangular. The basis for 2-eigenspace is \(\left\{ \begin{pmatrix}
    1 \\ 0
\end{pmatrix} \right\}\). The basis for -1-eigenspace is \(\left\{ \begin{pmatrix}
    -2 \\ 1
\end{pmatrix} \right\}\).

\begin{align}
    \begin{pmatrix}
        2 & 6 \\ 0 & -1
    \end{pmatrix} = \begin{pmatrix}
        -2 & 1 \\ 1 & 0
    \end{pmatrix} \begin{pmatrix}
        -1 & 0 \\ 0 & 2
    \end{pmatrix}
    \begin{pmatrix}
        -2 & 1 \\ 1 & 0
    \end{pmatrix}^{-1}
\end{align}

\noindent
\newline
\textbf{Example 2: Diagonalize if possible.}
\begin{equation}
    \begin{pmatrix}
        3 & 1 \\ 0 & 3
    \end{pmatrix}
\end{equation}

\noindent
\(\lambda_1 = 3\), \(\lambda_2 = 3\) because upper triangular. The basis for 3-eigenspace is \(\left\{ \begin{pmatrix}
    1 \\ 0
\end{pmatrix} \right\}\).

\begin{align}
    \begin{pmatrix}
        3 & 1 \\ 0 & 3
    \end{pmatrix} \ne \begin{pmatrix}
        1 & 1 \\ 0 & 0
    \end{pmatrix} \begin{pmatrix}
        3 & 0 \\ 0 & 3
    \end{pmatrix}
    \begin{pmatrix}
        1 & 1 \\ 0 & 0
    \end{pmatrix}^{-1}
\end{align}

\noindent
Not diagonalizable, since the \(\text{rank} \begin{pmatrix}
        1 & 1 \\ 0 & 0
    \end{pmatrix} \ne 2\). \(P^{-1}\) does not exist due to IMT.

\noindent
\newline
\textbf{Example 3: Diagonalize if possible.}
\begin{equation}
    \begin{pmatrix}
        7 & 4 & 16 \\ 2 & 5 & 8 \\ -2 & -2 & -5
    \end{pmatrix}
\end{equation}

\noindent
\(\lambda_1 = 3\), \(\lambda_2 = 1\) because upper triangular. 

\begin{align}
    A - I = \begin{pmatrix}
        6 & 4 & 16 \\ 2 & 4 & 8 \\ -2 & -2 & -6
    \end{pmatrix} \Rightarrow \begin{pmatrix}
        1 & 0 & 2 \\
        0 & 1 & 1 \\
        0 & 0 & 0
    \end{pmatrix}
\end{align}

\noindent
The basis for 1-eigenspace is \(\left\{ \begin{pmatrix}
    -2 \\ -1 \\ 1
\end{pmatrix} \right\}\). 

\begin{align}
    A - 3I = \begin{pmatrix}
        4 & 4 & 16 \\
        2 & 2 & 8 \\
        -2 & -2 & -8
    \end{pmatrix} \Rightarrow \begin{pmatrix}
        1 & 1 & 4 \\
        0 & 0 & 0 \\
        0 & 0 & 0
    \end{pmatrix}
\end{align}

\noindent
The basis for 3-eigenspace is \(\left\{ \begin{pmatrix}
    -1 \\ 1 \\ 0
\end{pmatrix} \begin{pmatrix}
    -4 \\ 0 \\ 1
\end{pmatrix} \right\}\).

\begin{align}
    \begin{pmatrix}
        7 & 4 & 16 \\ 2 & 5 & 8 \\ -2 & -2 & -5
    \end{pmatrix} = \begin{pmatrix}
        -2 & -1 & -4 \\ -1 & 1 & 0 \\ 1 & 0 & 1
    \end{pmatrix} \begin{pmatrix}
        1 & 0 & 0 \\ 0 & 3 & 3 \\ 0 & 0 & 3
    \end{pmatrix}
    \begin{pmatrix}
        -2 & -1 & -4 \\ -1 & 1 & 0 \\ 1 & 0 & 1
    \end{pmatrix}^{-1}
\end{align}

\subsection{Diagonalization Theorem}
\begin{theorem}
    An \(n\times n\) matrix \(A\) is diagonalizable if and only if \(A\) has \(n\) linearly independent eigenvectors. In fact, \(A=PDP^{-1}\), with \(D\) a diagonal matrix, if and only if the columns of \(P\) are \(n\) linearly independent eigenvectors of \(A\). In this case, the diagonal entries of \(D\) are eigenvalues of \(A\) that correspond, respectively, to the eigenvectors in \(P\).
\end{theorem}

\noindent
\newline
\textbf{Example 3: Diagonalize if possible. \(\lambda_1=1, \lambda_2=-2\)}
\begin{equation}
    \begin{pmatrix}
        2 & 4 & 2 \\ -4 & -6 & -3 \\ 3 & 3 & 1
    \end{pmatrix}
\end{equation}

\begin{align}
    A - I = \begin{pmatrix}
        1 & 4 & 3 \\ -4 & -7 & -3 \\ 3 & 3 & 0
    \end{pmatrix} \Rightarrow \begin{pmatrix}
        1 & 0 & -1 \\
        0 & 1 & 1 \\
        0 & 0 & 0
    \end{pmatrix}
\end{align}

\noindent
The basis for 1-eigenspace is \(\left\{ \begin{pmatrix}
    1 \\ -1 \\ 1
\end{pmatrix} \right\}\). 

\begin{align}
    A + 2I = \begin{pmatrix}
        4 & 4 & 3 \\
        -4 & -4 & -3 \\
        3 & 3 & 3
    \end{pmatrix} \Rightarrow \begin{pmatrix}
        1 & 1 & 0 \\
        0 & 0 & 1 \\
        0 & 0 & 0
    \end{pmatrix}
\end{align}

\noindent
The basis for -2-eigenspace is \(\left\{ \begin{pmatrix}
    -1 \\ 1 \\ 0
\end{pmatrix} \right\}\).

\begin{align}
    \begin{pmatrix}
        2 & 4 & 2 \\ -4 & -6 & -3 \\ 3 & 3 & 1
    \end{pmatrix} \ne \begin{pmatrix}
        1 & -1 & * \\
        -1 & 1 & * \\
        1 & 0 & *
    \end{pmatrix} \begin{pmatrix}
        1 & 0 & 0 \\
        0 & -2 & 0 \\
        0 & 0 & *
    \end{pmatrix}
    \begin{pmatrix}
        1 & -1 & * \\
        -1 & 1 & * \\
        1 & 0 & *
    \end{pmatrix}^{-1}
\end{align}

\noindent
Not diagonalizable, since the \(\text{rank} \begin{pmatrix}
        1 & -1 & * \\
        -1 & 1 & * \\
        1 & 0 & *
    \end{pmatrix} \ne 3\) (algebraic multiplicity for one eigenvalue is greater than the geometric multiplicity). \(P^{-1}\) does not exist due to IMT.

\subsection{Basis of Eigenvectors}
\textbf{Example 1:} Express the vector \(\Vec{x_0}=\begin{pmatrix}
    4 \\ 5
\end{pmatrix}\) as a linear combination of the vectors \(\Vec{v_1}=\begin{pmatrix}
    1 \\ 1
\end{pmatrix}\) and \(\Vec{v_2}= \begin{pmatrix}
    1 \\ -1
\end{pmatrix}\) and find the coordinates of \(\Vec{x_0}\) in the basis \(B=\{\Vec{v_1},\Vec{v_2}\}\).

\begin{equation}
    [\Vec{x_0}]_B = \begin{pmatrix}
        \frac{9}{2} \\ -\frac{1}{2}
    \end{pmatrix}
\end{equation}

\noindent
Let \(P=[\Vec{v_1}, \Vec{v_2}]\) and \(D=\begin{pmatrix}
    1 & 0 \\ 0 & -1
\end{pmatrix}\), and find \([A^k\Vec{x_0}]_B\), where \(A=PDP^{-1}\), for \(k=1,2,\dots\).
\begin{equation}
    A = \begin{pmatrix}
        0 & 1 \\ 1 & 0
    \end{pmatrix}
\end{equation}

\begin{equation}
    [A^k\Vec{x_0}]_B = \begin{pmatrix}
        \frac{9}{2} \\ (-1)^{k+1}\frac{1}{2}
    \end{pmatrix}
\end{equation}

\noindent
\newline
\textbf{Example 2:} Express the vector \(\Vec{x_0}=\begin{pmatrix}
    4 \\ 5
\end{pmatrix}\) as a linear combination of the vectors \(\Vec{v_1}=\begin{pmatrix}
    1 \\ 1
\end{pmatrix}\) and \(\Vec{v_2}= \begin{pmatrix}
    1 \\ -1
\end{pmatrix}\) and find the coordinates of \(\Vec{x_0}\) in the basis \(B=\{\Vec{v_1},\Vec{v_2}\}\).

\begin{equation}
    [\Vec{x_0}]_B = \begin{pmatrix}
        \frac{9}{2} \\ -\frac{1}{2}
    \end{pmatrix}
\end{equation}

\noindent
Let \(P=[\Vec{v_1}, \Vec{v_2}]\) and \(D=\begin{pmatrix}
    1 & 0 \\ 0 & -\frac{1}{2}
\end{pmatrix}\), and find \([A^k\Vec{x_0}]_B\), where \(A=PDP^{-1}\), for \(k=1,2,\dots\).
\begin{equation}
    A = \begin{pmatrix}
        0.25 & 0.75 \\ 0.75 & 0.25
    \end{pmatrix}
\end{equation}

\begin{equation}
    [A^k\Vec{x_0}]_B = \begin{pmatrix}
        \frac{9}{2} \\ (-\frac{1}{2})^{k+1}
    \end{pmatrix}
\end{equation}

\noindent
\newline
\textbf{Example 3:} Express the vector \(\Vec{x_0}=\begin{pmatrix}
    4 \\ 5
\end{pmatrix}\) as a linear combination of the vectors \(\Vec{v_1}=\begin{pmatrix}
    1 \\ 1
\end{pmatrix}\) and \(\Vec{v_2}= \begin{pmatrix}
    1 \\ -1
\end{pmatrix}\) and find the coordinates of \(\Vec{x_0}\) in the basis \(B=\{\Vec{v_1},\Vec{v_2}\}\).

\begin{equation}
    [\Vec{x_0}]_B = \begin{pmatrix}
        \frac{9}{2} \\ -\frac{1}{2}
    \end{pmatrix}
\end{equation}

\noindent
Let \(P=[\Vec{v_1}, \Vec{v_2}]\) and \(D=\begin{pmatrix}
    2 & 0 \\ 0 & \frac{3}{2}
\end{pmatrix}\), and find \([A^k\Vec{x_0}]_B\), where \(A=PDP^{-1}\), for \(k=1,2,\dots\).
\begin{equation}
    A = \begin{pmatrix}
        1.75 & 0.25 \\ 0.25 & 1.75
    \end{pmatrix}
\end{equation}

\begin{equation}
    [A^k\Vec{x_0}]_B = \begin{pmatrix}
        \frac{9}{2} \cdot 2^k \\ -\frac{1}{2}(\frac{3}{2})^k
    \end{pmatrix}
\end{equation}
\section{Complex Eigenvalues}
\begin{theorem}
\begin{enumerate} Complex Eigenvalues
    \item If \(\lambda \in \mathbb{C}\) is a root of a real polynomial \(p(x)\), then the conjugate \(\lambda\) is also a root of \(p(x)\).
    \item If \(\lambda\) is an eigenvalue of real matrix \(A\) with eigenvector \(\Vec{v}\), then \(\Bar{\lambda}\) is an eigenvalue of \(A\) with eigenvector \(\Vec{v}\).
\end{enumerate}
\end{theorem}

\section{Rotation-Dilation Matrix}
\begin{align}
    A &= \begin{pmatrix}
        a & -b \\ b & a
    \end{pmatrix} \\
    &= \begin{pmatrix}
        r & 0 \\ 0 & r
    \end{pmatrix}
    \begin{pmatrix}
        cos(\theta) & -sin(\theta) \\ sin(\theta) & cos(\theta)
    \end{pmatrix}
\end{align}

\noindent
The eigenvalues of \(A\) are \(a \pm bi\).
\section{Inner Product and Orthogonality}
\subsection{Dot Products and Orthogonality}
\begin{definition} Dot Product

    \[\Vec{u}\cdot \Vec{v} = \Vec{u}^T\Vec{v} = u_1 v_1 + u_2 v_2 + \dots + u_n v_n\]    
\end{definition}
\begin{definition} Orthogonality

    Two vectors \(\Vec{u}\) and \(\Vec{w}\) are orthogonal if \(\Vec{u} \cdot \Vec{v} = 0\). This is equivalent to:
    \[||\Vec{u}+\Vec{w}||^2=||\Vec{u}||^2+||\Vec{w}||^2\]
\end{definition}

\subsection{Orthogonal Compliments and Sets}
\begin{definition} Orthogonal Compliments

    Let \(W\) be a subspace of \(\Re^n\). Vector \(\Vec{z} \in \Re^n\) is orthogonal to \(W\) if \(\Vec{z}\) is orthogonal to every vector in \(W\).

    The set of all vectors orthogonal to \(W\) is a subspace, the orthogonal complement of \(W\) or \(W^\perp\).
    \[W^\perp = \{\Vec{x} \in \Re^n : \Vec{z} \cdot \Vec{w} = 0 \text{ for all } \Vec{w} \in W\}\]
\end{definition}
\begin{definition} Row Space

    \(\text{Row} \; A\) is the space spanned by the rows of matrix \(A\).
\end{definition}

\begin{definition}
    A set of vectors \(\{\Vec{u_1}, \dots, \Vec{u_p}\}\) are an orthogonal set of vectors if for each \(j\ne k, \Vec{u_j}\perp \Vec{v_k}\).
\end{definition}

\subsection{Projections}
\begin{definition} Projections

    \[\text{proj}_{\Vec{u}} \Vec{v} = \frac{\Vec{v} \cdot \Vec{u}}{\Vec{u} \cdot \Vec{u}}\Vec{u}\]
\end{definition}

\subsection{Inverse of Orthonormal Matrix}
\begin{theorem} Inverse of Orthonormal Matrix

The inverse of an orthonormal matrix \(A\) is \(A^{-1}=A^T\). In other words, \(A A^T = I\).
\end{theorem}

\subsection{Orthogonal Decomposition Theorem}
\begin{theorem}
    Let \(W\) be a subspace of \(\Re^n\). Then, each vector \(\Vec{y}\in \Re^n\) has the unique decomposition:
    \[\Vec{y}=\hat{y}+w^\perp, \; \hat{y}\in W, \; w^\perp \in W^\perp\]
    And, if \(\Vec{u_1},\dots,\Vec{u_p}\) is any orthogonal basis for \(W\).
    \[\hat{y}=\frac{\Vec{y}\Vec{u_1}}{\Vec{u_1}\Vec{u_1}}\Vec{u_1} + \dots + \frac{\Vec{y}\Vec{u_p}}{\Vec{u_p}\Vec{u_p}}\Vec{u_p}\]
    We say that \(\hat{y}\) is the orthogonal projection of \(\Vec{y}\) onto \(W\).
\end{theorem}

\subsection{Best Approximation Theorem}
\begin{theorem}
    Let \(W\) be a subspace of \(\Re^n\), \(\Vec{y}\in\Re^n\), and \(\hat{y}\) is the orthogonal projection of \(\Vec{y}\) onto \(W\). Then for any \(\Vec{w}\ne \hat{y} \in W\), we have:
    \[||\Vec{y}-\hat{y}|| < ||\Vec{y}-\Vec{w}||\]
    That is, \(\hat{y}\) is the unique vector in \(W\) that is closest to \(\Vec{y}\).
\end{theorem}
\section{Gram-Schmidt Process}
\begin{theorem} The Gram-Schmidt Process

    Given a basis \(\{\bold{x_1}, \dots, \bold{x_p}\}\) for a nonzero subspace \(W\) of \(\Re^n\), define:
    \[\bold{v_1}=\bold{x_1}\]
    \[\bold{v_2}=\bold{x_2}-\frac{\bold{x_2}\cdot \bold{v_1}}{\bold{v_1} \cdot \bold{v_1}}\bold{v_1}\]
    \[\bold{v_3}=\bold{x_3}-\frac{\bold{x_3}\cdot \bold{v_1}}{\bold{v_1} \cdot \bold{v_1}}\bold{v_1}-\frac{\bold{x_3}\cdot \bold{v_2}}{\bold{v_2} \cdot \bold{v_2}}\bold{v_2}\]
    \[\vdots \notag\]
    \[\bold{v_p}=\bold{x_p}-\frac{\bold{x_p}\cdot \bold{v_1}}{\bold{v_1} \cdot \bold{v_1}}\bold{v_1}-\frac{\bold{x_p}\cdot \bold{v_2}}{\bold{v_2} \cdot \bold{v_2}}\bold{v_2} - \dots - \frac{\bold{x_p}\cdot \bold{v_{p-1}}}{\bold{v_{p-1}} \cdot \bold{v_{p-1}}}\bold{v_{p-1}}\]

    Then \(\{\bold{v_1},\dots,\bold{v_p}\}\) is an orthogonal basis for \(W\). In addition,
    \[\text{Span} \{\bold{v_1},\dots,\bold{v_p}\} = \text{Span} \{\bold{x_1},\dots,\bold{x_p}\}\; \text{ for } 1 \le k \le p\]
\end{theorem}

\begin{definition}
    A set of vectors from an orthonormal basis if the vectors are mutually orthogonal and have unit length.
\end{definition}

\subsection{QR Factorization}
\begin{theorem} QR Factorization

Any \(m\times n\) matrix \(A\) with linearly independent columns has the QR factorization:
\[A = QR\]

\noindent
where:

\begin{enumerate}
    \item \(Q\) is \(m\times n\), its columns are an orthonormal basis for \(\text{Col} \; A\).
    \item \(R\) is \(n\times n\), upper triangular, with positive entries on its diagonal, and the length of the jth column of \(R\) is equal to the length of the jth column of \(A\).
\end{enumerate}
\end{theorem}

\noindent
\newline
\textbf{Example 1: }
\begin{equation}
    \Vec{v_1} = \begin{pmatrix}
        3 \\ 2 \\ 0
    \end{pmatrix}, \; \Vec{v_2} = \begin{pmatrix}
        -2 \\ 3 \\ 1
    \end{pmatrix}
\end{equation}

\noindent
\textbf{Steps:}
\begin{enumerate}
    \item Determine \(Q\) matrix.
    \begin{enumerate}
        \item Do Gram-Schmidt to the columns of \(A\). In this case, the columns of \(A\) are already orthogonal.
        \item Normalize the basis.
    \end{enumerate}
    \item Calculate \(R = Q^T A\).
\end{enumerate}

\begin{equation}
    \{\Vec{w_1}, \Vec{w_2}\} = \left\{ \begin{pmatrix}
        \frac{3}{\sqrt{13}} \\ \frac{2}{\sqrt{13}} \\ 0
    \end{pmatrix}, \; \begin{pmatrix}
        \frac{-2}{\sqrt{14}} \\ \frac{3}{\sqrt{14}} \\ \frac{1}{\sqrt{14}}
    \end{pmatrix} \right\}
\end{equation}

\begin{align}
    Q = \begin{pmatrix}
        \frac{3}{\sqrt{13}} & \frac{-2}{\sqrt{14}} \\
        \frac{2}{\sqrt{13}} & \frac{3}{\sqrt{14}} \\
        0 & \frac{1}{\sqrt{14}} \\
    \end{pmatrix}
\end{align}

\begin{align}
    R &= \begin{pmatrix}
        \frac{3}{\sqrt{13}} & \frac{2}{\sqrt{13}} & 0 \\
        \frac{-2}{\sqrt{14}} & \frac{3}{\sqrt{14}} & \frac{1}{\sqrt{14}}
    \end{pmatrix}
    \begin{pmatrix}
        3 & -2 \\ 2 & 3 \\ 0 & 1
    \end{pmatrix}\\
    &= \begin{pmatrix}
        \sqrt{13} & 0 \\
        0 & \sqrt{14}
    \end{pmatrix}
\end{align}
\newline
\section{Least-Squares Problems}
\begin{definition}
    If \(A\) is \(m\times n\) and \(\Vec{b}\) is in \(\Re^m\), a least-squares solution of \(A\Vec{x}= \Vec{b}\) is an \(\hat{x}\) in \(\Re^n\) such that:
    \[||\Vec{b} - A\hat{x}|| \le ||\Vec{b}-A\Vec{x}||\]

    \noindent
    for all \(\Vec{x}\) in \(\Re^n\).
\end{definition}

\noindent
\newline
\textbf{Example 1: Find a least-squares solution of the inconsistent system \(A\Vec{x} = \Vec{b}\) for:}
\begin{align}
    A = \begin{pmatrix}
        4 & 0 \\ 0 & 2 \\ 1 & 1
    \end{pmatrix}, \; \; \;
    \Vec{b} = \begin{pmatrix}
        2 \\ 0 \\ 11
    \end{pmatrix}
\end{align}

\begin{enumerate}
    \item Compute \(A^T A\) and \(A^T \Vec{b}\)
    \item Solve \(A^T A \hat{x}=A^T \Vec{b}\)
\end{enumerate}

\begin{align}
    A^T A &= \begin{pmatrix}
        4 & 0 & 1 \\ 0 & 2 & 1
    \end{pmatrix} \begin{pmatrix}
        4 & 0 \\ 0 & 2 \\ 1 & 1 
    \end{pmatrix} = \begin{pmatrix}
        17 & 1 \\ 1 & 5
    \end{pmatrix} \\
    A^T \Vec{b} &= \begin{pmatrix}
        4 & 0 & 1 \\ 0 & 2 & 1
    \end{pmatrix} \begin{pmatrix}
        2 \\ 0 \\ 11
    \end{pmatrix} = \begin{pmatrix}
        19 \\ 11
    \end{pmatrix}
\end{align}

\begin{align}
    [A^T A | A^T \Vec{b}] &= \begin{pmatrix}[cc|c]
        17 & 1 & 19 \\ 1 & 5 & 11
    \end{pmatrix} \\
    &\Rightarrow \begin{pmatrix}[cc|c]
        1 & 0 & 1 \\
        0 & 1 & 2
    \end{pmatrix} \\
    \hat{x} = \begin{pmatrix}
        1 \\ 2
    \end{pmatrix}
\end{align}

\noindent
\(\hat{x}\) provides the weights to multiply by the columns of \(A\) in order to find the projection of \(\Vec{b}\).

\noindent
\newline
\textbf{Example 2: Fit a line \(y=\alpha x+\beta\) that best fits the data (0,0.5), (1,1), (2,2.5), (3,3).}

\begin{enumerate}
    \item Plug in data into the model.
    \item Compute \(A^T A\) and \(A^T \Vec{b}\)
    \item Solve \(A^T A \hat{x}=A^T \Vec{b}\)
\end{enumerate}

\begin{align}
    0.5 &= \alpha \cdot 0 + \beta \\
    1 &= \alpha \cdot 1 + \beta \\
    2.5 &= \alpha \cdot 2 + \beta \\
    3 &= \alpha \cdot 3 + \beta
\end{align}

\begin{align}
    A = \begin{pmatrix}
        0 & 1 \\ 1 & 1 \\ 2 & 1 \\ 3 & 1
    \end{pmatrix}, \; \; \;
    \Vec{b} = \begin{pmatrix}
        0.5 \\ 1 \\ 2.5 \\ 3
    \end{pmatrix}, \; \; \;
    \Vec{x} = \begin{pmatrix}
        \alpha \\ \beta
    \end{pmatrix}
\end{align}

\begin{align}
    A^T A &= \begin{pmatrix}
        0 & 1 & 2 & 3 \\ 1 & 1 & 1 & 1
    \end{pmatrix} \begin{pmatrix}
        0 & 1 \\ 1 & 1 \\ 2 & 1 \\ 3 & 1
    \end{pmatrix} = \begin{pmatrix}
        14 & 6 \\ 6 & 4
    \end{pmatrix} \\
    A^T \Vec{b} &= \begin{pmatrix}
        0 & 1 & 2 & 3 \\ 1 & 1 & 1 & 1
    \end{pmatrix} \begin{pmatrix}
        0.5 \\ 1 \\ 2.5 \\ 3
    \end{pmatrix} = \begin{pmatrix}
        15 \\ 7
    \end{pmatrix}
\end{align}

\begin{align}
    A^T A \hat{x} &= A^T \Vec{b} \\
    \hat{x} &= (A^T A)^{-1} A^T \Vec{b} \\
    &= \frac{1}{20} \begin{pmatrix}
        4 & -6 \\
        -6 & 14
    \end{pmatrix} \begin{pmatrix}
        15 \\ 7
    \end{pmatrix} \\
    &= \begin{pmatrix}
        0.9 \\ 0.4
    \end{pmatrix}
\end{align}

\noindent
The Sum of Squared Errors (SSE) \(||\Vec{b}-\hat{b}||^2=\sum_{i=1}^k (\Vec{b}_i - \hat{b}_i)^2\) is minimized. \(\frac{1}{k}||\Vec{b}-\hat{b}||^2=\frac{1}{k}\sum_{i=1}^k (\Vec{b}_i - \hat{b}_i)^2\) is equivalent to the Mean Squared Error (MSE) of the line.

\subsection{Proof of Least Squares}
\begin{align}
    \Vec{b} - \hat{b} \in (\text{Col}\; A)^\perp &= \text{Nul} \; A^T \\
    A^T (\Vec{b} - \hat{b}) &= \Vec{0} \\
    A^T \Vec{b} - A^T \hat{b} &= \Vec{0} \\
    A^T \Vec{b} - A^T A \Vec{x} &= \Vec{0} \\
    A^T A \Vec{x} &= A^T \Vec{b}
\end{align}

\subsection{Using Least Squares for Complex Models}
\textbf{Example 1: Fit a line \(y=\alpha x^2+\beta x + \gamma\) that best fits the data (0,1), (1,1), (2,3).}

\begin{enumerate}
    \item Plug in data into the model.
    \item Compute \(A^T A\) and \(A^T \Vec{b}\)
    \item Solve \(A^T A \hat{x}=A^T \Vec{b}\)
\end{enumerate}

\begin{align}
    1 &= \alpha \cdot 0 + \beta \cdot 0 + \gamma \\
    1 &= \alpha \cdot 1 + \beta \cdot 1 + \gamma \\
    3 &= \alpha \cdot 4 + \beta \cdot 2 + \gamma
\end{align}

\begin{align}
    A = \begin{pmatrix}
        0 & 0 & 1 \\ 1 & 1 & 1 \\ 4 & 2 & 1
    \end{pmatrix}, \; \; \;
    \Vec{b} = \begin{pmatrix}
        1 \\ 1 \\ 3
    \end{pmatrix}, \; \; \;
    \Vec{x} = \begin{pmatrix}
        \alpha \\ \beta \\ \gamma
    \end{pmatrix}
\end{align}

\begin{align}
    A^T A &= \begin{pmatrix}
        0 & 1 & 4 \\ 0 & 1 & 2 \\ 1 & 1 & 1
    \end{pmatrix} \begin{pmatrix}
        0 & 0 & 1 \\ 1 & 1 & 1 \\ 4 & 2 & 1
    \end{pmatrix} = \begin{pmatrix}
        17 & 9 & 5 \\
        9 & 5 & 3 \\
        5 & 3 & 3
    \end{pmatrix} \\
    A^T \Vec{b} &= \begin{pmatrix}
        0 & 1 & 4 \\ 0 & 1 & 2 \\ 1 & 1 & 1
    \end{pmatrix} \begin{pmatrix}
        1 \\ 1 \\ 3
    \end{pmatrix} = \begin{pmatrix}
        13 \\ 7 \\ 5
    \end{pmatrix}
\end{align}

\begin{align}
    A^T A \hat{x} &= A^T \Vec{b} \\
    \hat{x} &= (A^T A)^{-1} A^T \Vec{b} \\
    &= \begin{pmatrix}
        \frac{1}{2} & -1 & \frac{1}{2} \\
        -\frac{3}{2} & 2 & -\frac{1}{2} \\
        1 & 0 & 0
    \end{pmatrix} \begin{pmatrix}
        13 \\ 7 \\ 5
    \end{pmatrix} \\
    &= \begin{pmatrix}
        1 \\ -1 \\ 1
    \end{pmatrix}
\end{align}

\subsection{Unique Solutions for Least Squares}
\begin{theorem}
    Let \(A\) be any \(m\times n\) matrix. These statements are equivalent.
    \begin{enumerate}
        \item The equation \(A\Vec{x} = \Vec{b}\) has a unique least-squares solution for each \(\Vec{b} \in \Re^m\).
        \item The columns of \(A\) are linearly independent.
        \item The matrix \(A^T A\) is invertible.
    \end{enumerate}
    And, if these statements hold, the least square solution is:
    \[\hat{x} = (A^T A)^{-1} A^T \Vec{b}\]
\end{theorem}

\subsection{Least Squares and QR Decomposition}
\begin{theorem} Least Squares and QR Decomposition

    Let \(m\times n\) matrix \(A\) have a QR decomposition. Then for each \(\Vec{b} \in \Re^m\) the equation \(A \Vec{x} = \Vec{b}\) has a unique least squares solution
    \[R \hat{x} = Q^T \Vec{b}\]
    (Remember, R is upper triangular, so the equation above is solved by back-substitution.)
\end{theorem}
\section{Finite State Markov Chains}
\subsection{Steady-State Vector and Google Page Rank}
The steady state vector can be found by applying \(P\Vec{q}=\Vec{q}\), which means that the row is reduced \(P-I\). By definition, the steady state vector \(\Vec{q}\) will have an eigenvalue of 1.

\noindent
If \(P\) is regular, then \(P^n \Vec{e_i}=\Vec{q}\). Therefore:

\begin{equation}
    P = \begin{pmatrix}
        \Vec{q} & \Vec{q} & \dots & \Vec{q}
    \end{pmatrix}
\end{equation}

\begin{theorem} Google Page Rank

If \(P\) is a regular \(m \times m\) transition matrix with \(m \ge 2\), then the following statements are all true.
\begin{enumerate}
    \item There is a stochastic matrix \(\Pi\) such that:
    \[\lim_{n\rightarrow\infty} P^n=\Pi\]
    \item Each column of \(\Pi\) is the same probability vector \(\Vec{q}\).
    \item For any initial probability vector \(\Vec{x_0}\),
    \[\lim_{n\rightarrow\infty} P^n \Vec{x_0} = \Vec{q}\]
    \item \(P\) has a unique eigenvector, \(\Vec{q}\), which has eigenvalue \(\lambda=1\).
    \item The eigenvalues of \(P\) satisfy \(|\lambda| \le 1\).
\end{enumerate}
    
\end{theorem}

\noindent
\newline
\textbf{Example 1:}
\begin{enumerate}
    \item Generate the Google Matrix.
    \item Adjust matrix to fix columns that do not sum to 1 (aka have no links).
    \item Use damping factor to adjust for user behavior.
    \item Solve \(G\Vec{q}=\Vec{q}\).
\end{enumerate}
\begin{align}
    P &= \begin{pmatrix}
        0 & 0.5 & 0.5 & 0 & 0 \\
        0.5 & 0 & 0.5 & 0.5 & 0 \\
        0 & 0.5 & 0 & 0 & 0 \\
        0.5 & 0 & 0 & 0 & 0 \\
        0 & 0 & 0 & 0.5 & 0
    \end{pmatrix} \\
    P^* &= \begin{pmatrix}
        0 & 0.5 & 0.5 & 0 & 0.2 \\
        0.5 & 0 & 0.5 & 0.5 & 0.2 \\
        0 & 0.5 & 0 & 0 & 0.2 \\
        0.5 & 0 & 0 & 0 & 0.2 \\
        0 & 0 & 0 & 0.5 & 0.2
    \end{pmatrix}
\end{align}
\begin{align}
    K &= \begin{pmatrix}
        0.2 & 0.2 & 0.2 & 0.2 & 0.2 \\
        0.2 & 0.2 & 0.2 & 0.2 & 0.2 \\
        0.2 & 0.2 & 0.2 & 0.2 & 0.2 \\
        0.2 & 0.2 & 0.2 & 0.2 & 0.2 \\
        0.2 & 0.2 & 0.2 & 0.2 & 0.2 \\
    \end{pmatrix} \\
    G &= 0.85 P^* + 0.15 K \\
    \Vec{q} &= \begin{pmatrix}
        0.251 \\ 0.298 \\ 0.1766 \\ 0.156 \\ 0.116
    \end{pmatrix}
\end{align}

\noindent
Therefore, the page ranking is B, A, C, D, E.

\noindent
\newline
\textbf{Example 2:}
\begin{enumerate}
    \item Generate the Google Matrix.
    \item Adjust matrix to fix columns that do not sum to 1 (aka no links).
    \item Use damping factor to adjust for user behavior.
    \item Solve \(G\Vec{q}=\Vec{q}\).
\end{enumerate}
\begin{align}
    P &= \begin{pmatrix}
        0 & 0 & 1 & 0 & 0 \\
        \frac{1}{3} & 0 & 0 & \frac{1}{2} & 0 \\
        \frac{1}{3} & 0 & 0 & \frac{1}{2} & 0 \\
        \frac{1}{3} & \frac{1}{2} & 0 & 0 & 0 \\
        0 & \frac{1}{2} & 0 & 0 & 0
    \end{pmatrix} \\
    P^* &= \begin{pmatrix}
        0 & 0 & 1 & 0 & \frac{1}{5} \\
        \frac{1}{3} & 0 & 0 & \frac{1}{2} & \frac{1}{5} \\
        \frac{1}{3} & 0 & 0 & \frac{1}{2} & \frac{1}{5} \\
        \frac{1}{3} & \frac{1}{2} & 0 & 0 & \frac{1}{5} \\
        0 & \frac{1}{2} & 0 & 0 & \frac{1}{5}
    \end{pmatrix}
\end{align}
\begin{align}
    K &= \begin{pmatrix}
        0.2 & 0.2 & 0.2 & 0.2 & 0.2 \\
        0.2 & 0.2 & 0.2 & 0.2 & 0.2 \\
        0.2 & 0.2 & 0.2 & 0.2 & 0.2 \\
        0.2 & 0.2 & 0.2 & 0.2 & 0.2 \\
        0.2 & 0.2 & 0.2 & 0.2 & 0.2 \\
    \end{pmatrix} \\
    G &= 0.85 P^* + 0.15 K \\
    \Vec{q} &= \begin{pmatrix}
        0.231 \\ 0.208 \\ 0.208 \\ 0.208 \\ 0.142
    \end{pmatrix}
\end{align}

\noindent
Therefore, the page ranking is A, (B, C, D), E.

\section{Symmetric Matrices}
\subsection{Definition}
\begin{definition} Definition of Symmetric Matrices

    Matrix \(A\) is symmetric if \(A^T = A\).
\end{definition}

\subsection{Symmetry of \(A A^T\)}
\(A A^T\) is also a symmetric matrix:
\begin{align}
    A A^T &= \begin{pmatrix}
        \text{---} & a_1 & \text{---} \\
        \text{---} & a_2 & \text{---} \\
        & \vdots & \\
        \text{---} & a_n & \text{---}
    \end{pmatrix}
    \begin{pmatrix}
        \spike{25pt}{$a_1$}  & \spike{25pt}{$a_2$} & \dots & \spike{25pt}{$a_n$}
    \end{pmatrix} \\
    &= \begin{pmatrix}
        a_1 \cdot a_1 & \dots  & a_1 \cdot a_n \\
        \vdots & \ddots & \vdots \\
        a_n \cdot a_1 & \dots  & a_n \cdot a_n
    \end{pmatrix}
\end{align}

\subsection{Eigenspaces of Symmetric Matrices}
\begin{theorem}
    \(A\) is a symmetric matrix, with eigenvectors \(\Vec{v_1}\) and \(\Vec{v_2}\) corresponding to two distinct eigenvalues. Then, \(\Vec{v_1}\) and \(\Vec{v_2}\) are orthogonal. More generally, eigenspaces associated with distinct eigenvalues are orthogonal subspaces.
\end{theorem}

\subsection{Spectral Theorem}
\begin{theorem}
    An \(n \times n\) symmetric matrix \(A\) has the following properties:
    \begin{enumerate}
        \item All eigenvalues of \(A\) are real.
        \item The dimension of each eigenspace is full, that its dimension is equal to its algebraic multiplicity.
        \item The eigenspace are mutually orthogonal.
        \item \(A\) can be diagonalized: \(A = P D P^{-1} = P D P^T\), where \(D\) is diagonal and \(P\) is an orthogonal matrix.
    \end{enumerate}
\end{theorem}

\begin{definition} Spectral Decomposition

    Suppose \(A\) can be diagonalized as \(A = P D P^T\), then \(A\) has the decomposition
    \[A = \lambda_1 U_1 U_1^T + \lambda_2 U_2 U_2^T + \dots + \lambda_n U_n U_n^T = \sum_{k=1}^{n} \lambda_k U_k U_k^T \]
    The following is true about the spectral decomposition:
    \begin{enumerate}
        \item \(U_n U_n^T\) will always be rank 1. 
        \item \(U_1 U_1^T \Vec{x} = \text{proj}_{U_1} (\Vec{x})\)
        \item After ordering the terms largest to smallest by their corresponding eigenvalue, one can determine a rank \(n\) approximation of \(A\) by taking the first \(n\) terms.
    \end{enumerate}
\end{definition}

\noindent
\newline
\textbf{Example:}
\begin{align}
    A &= \begin{pmatrix}
        2 & 0 & 2 \\
        0 & 1 & 0 \\
        2 & 0 & 2
    \end{pmatrix} \\
    &= \lambda_1 U_1 U_1^T + \lambda_2 U_2 U_2^T + \lambda_3 U_3 U_3^T \\
    &= 4 \begin{pmatrix}
        \frac{1}{\sqrt{2}} \\ 0 \\ \frac{1}{\sqrt{2}}
    \end{pmatrix} \begin{pmatrix}
        \frac{1}{\sqrt{2}} & 0 & \frac{1}{\sqrt{2}}
    \end{pmatrix} + 1 \begin{pmatrix}
        0 \\ 1 \\ 0 
    \end{pmatrix} \begin{pmatrix}
        0 & 1 & 0 
    \end{pmatrix} + 0 \begin{pmatrix}
        -\frac{1}{\sqrt{2}} \\ 0 \\ \frac{1}{\sqrt{2}}
    \end{pmatrix} \begin{pmatrix}
        -\frac{1}{\sqrt{2}} & 0 & \frac{1}{\sqrt{2}}
    \end{pmatrix} \\
    &= 4 \begin{pmatrix}
        \frac{1}{2} & 0 & \frac{1}{2} \\
        0 & 0 & 0 \\
        \frac{1}{2} & 0 & \frac{1}{2}
    \end{pmatrix} + 1 \begin{pmatrix}
        0 & 0 & 0 \\
        0 & 1 & 0 \\
        0 & 0 & 0
    \end{pmatrix}
\end{align}
\section{Quadratic Forms}
\subsection{Definition}
\begin{definition}
    A quadratic form is a function \(Q: \Re^n \rightarrow \Re\), given by
    \[Q(\Vec{x}) = \Vec{x}^T A \Vec{x}\]
\end{definition}

\subsection{Converting between Quadratic Forms}
\textbf{Example 1:}
\begin{align}
    Q_A(x_1, x_2) = \Vec{x}^T A \Vec{x} &= \begin{pmatrix}
        x_1 x_2
    \end{pmatrix} \begin{pmatrix}
        4 & 0 \\ 0 & 3
    \end{pmatrix} \begin{pmatrix}
        x_1 \\ x_2
    \end{pmatrix} \\ &= 4x_1^2 + 3x_2^2 \\
    Q_B(x_1, x_2) = \Vec{x}^T A \Vec{x} &= \begin{pmatrix}
        x_1 x_2
    \end{pmatrix} \begin{pmatrix}
        4 & 1 \\ 1 & -3
    \end{pmatrix} \begin{pmatrix}
        x_1 \\ x_2
    \end{pmatrix} \\ &= 4x_1^2 + 2x_1 x_2 - 3x_2^2
\end{align}

\noindent
\newline
\textbf{Example 2: Write \(Q\) in the form \(\Vec{x}^T A \Vec{x}\).}
\begin{align}
    Q_A(x_1, x_2) = \Vec{x}^T A \Vec{x} &= 5x_1^2 - x_2^2 + 3x_3^2 + 6x_1 X_3 - 12 x_2 x_3 \\
    &= \begin{pmatrix}
        x_1 x_2 x_3
    \end{pmatrix} \begin{pmatrix}
        5 & 0 & 3 \\
        0 & -1 & -6 \\
        3 & -6 & 3
    \end{pmatrix} \begin{pmatrix}
        x_1 \\ x_2 \\ x_3
    \end{pmatrix} \\
\end{align}

\subsection{Change of Variables}
\textbf{Procedure to find \(\Vec{y} = P^T \Vec{x}\):}
\begin{align}
    A &= PDP^T \\
    Q_A(\Vec{x}) &= \Vec{x}^T A \Vec{x} \\
    &= \Vec{x}^T PDP^T \Vec{x} \\
    &= (P^T \Vec{x})^T D P^T \Vec{x} \\
    &= \Vec{y}^T D \Vec{y} \\
    &= Q_D(\Vec{y})
\end{align}
In order to remove cross-terms from the quadratic form, one can define \(Q_A(\Vec{x}) = Q_D(\Vec{y}) = \Vec{y}^T D \Vec{y}\). \(P^T\) is a rotation matrix that transforms the quadratic form such that the cross-terms when written as a function of \(\Vec{x}\) are removed when written as a function of \(\Vec{y}=P^T \Vec{x}\).

\noindent
\newline
\textbf{Example 1:}
\begin{align}
    A &= \begin{pmatrix}
        3 & 2 \\ 2 & 6
    \end{pmatrix} = PDP^T \\
    P &= \frac{1}{\sqrt{5}} \begin{pmatrix}
        2 & 1 \\ -1 & 2
    \end{pmatrix} \\
    D &= \begin{pmatrix}
        2 & 0 \\ 0 & 7
    \end{pmatrix} \\
    Q_A(x_1, x_2) &= 3x_1^2 + 4x_1 x_2 + 6x_2^2 \\
    Q_D(y_1, y_2) &= 2y_1^2 + 7y_2^2 \\
    \Vec{y} &= P^T \Vec{x} \\
    \begin{pmatrix}
        y_1 \\ y_2
    \end{pmatrix} &= \begin{pmatrix}
        \frac{2}{\sqrt{5}} x_1 -\frac{1}{\sqrt{5}} x_2 \\
        \frac{1}{\sqrt{5}} x_1 +\frac{2}{\sqrt{5}} x_2
    \end{pmatrix}
\end{align} 

\begin{theorem} Principal Axes Theorem

    If \(A\) is a symmetric matrix then there exists an orthogonal change of variable \(\Vec{x}=P\Vec{y}\) that transforms \(\Vec{x}^T A \Vec{x}\) to \(\Vec{y}^T D \Vec{y}\) with no cross-product terms.
\end{theorem}

\subsection{Classifying Quadratic Forms}
\begin{definition}
    A quadratic form \(Q\) is:
    \begin{enumerate}
        \item positive definite if \(Q(\Vec{x}) > 0\) for all \(\Vec{x}\ne\Vec{0}\).
        \item negative definite if \(Q(\Vec{x}) < 0\) for all \(\Vec{x}\ne\Vec{0}\).
        \item positive semidefinite if \(Q(\Vec{x}) \ge 0\) for all \(\Vec{x}\).
        \item negative semidefinite if \(Q(\Vec{x}) \le 0\) for all \(\Vec{x}\).
        \item indefinite if \(Q(\Vec{x}) \in \Re\) for all \(\Vec{x}\).
    \end{enumerate}
\end{definition}

\subsection{Quadratic Forms and Eigenvalues}
\begin{theorem}
    If \(A\) is a symmetric matrix with eigenvalues \(\lambda_i\), then \(Q = \Vec{x}^T A \Vec{x}\) is:
    \begin{enumerate}
        \item positive definite iff \(\lambda_i > 0\).
        \item negative definite iff \(\lambda_i < 0\).
        \item positive semidefinite iff \(\lambda_i \ge 0\).
        \item negative semidefinite iff \(\lambda_i \le 0\).
        \item indefinite iff \(\lambda_i > 0\) and some other \(\lambda_i < 0\).
    \end{enumerate}
\end{theorem}
\section{Constrained Optimization}
\subsection{Quadratic Forms}
\textbf{Example 1: Find the largest and smallest of \(Q(\Vec{x}) = 9 x_1^2 + 4 x_2^2 + 3 x_3^2 \) on the surface of an unit sphere.}

\begin{align}
    \text{Max: }& Q(1, 0, 0) = 9 \\
    \text{Min: }& Q(0, 0, 1) = 3
\end{align}

\noindent
\newline
If \(Q(\Vec{x}) = a x_1^2 + b x_2^2 + c x_3^2\) and \(||v|| = 1\), then \(Q(\Vec{v}\) is a weighted average of \(a\), \(b\), \(c\).

\subsection{Using Eigenvalues for Constrained Optimization}
\noindent
\newline
\textbf{Example 1: Find the largest and smallest of \(Q(\Vec{x}) = 3 x_1^2 + 7 x_2^2 \) on the edge of an unit circle.}

\begin{align}
    \text{Max: }& Q(0, 1) = 7 \\
    \text{Min: }& Q(1, 0) = 3
\end{align}

\noindent
\newline
\textbf{Example 3: } Let \(A = \begin{pmatrix}
    3 & 2 & 1 \\ 2 & 3 & 1 \\ 1 & 1 & 4
\end{pmatrix}\). Find the maximum value of the quadratic form \(\Vec{x^T A \Vec{x}}\) subject to the constraint \(\Vec{x}^T \Vec{x} = 1\) and find an unit vector at which this maximum value is attained.

\begin{align}
    Q(x_1, x_2, x_3) &= 3 x_1^2 + 3 x_2^2 + 4 x_3^2 + 4 x_1 x_2 + 2 x_1 x_3 + 2 x_2 x_3 \\
    p(\lambda) &= -(\lambda-6)(\lambda-3)(\lambda-1) = 0 \\
    \lambda_1 = 6, \lambda_2 &= 3, \lambda_3 = 1 \\
    Q_D &= 6y_1^2 + 3y_2^2 + y_3^2 \\
    \text{Max: }& Q_D(1, 0, 0) = 6
\end{align}

\begin{align}
    P \Vec{y} &= \Vec{x} \Leftarrow \text{Unit Length Eigenvector in 6-eigenspace} \\
    A-6I = \begin{pmatrix}
        -3 & 2 & 1 \\ 2 & -3 & 1 \\ 1 & 1 & -2
    \end{pmatrix}
    &\rightarrow \begin{pmatrix}
        1 & 1 & -2 \\ 0 & 5 & -5 \\ 0 & -5 & 5
    \end{pmatrix} \rightarrow \begin{pmatrix}
        1 & 0 & -1 \\ 0 & 1 & -1 \\ 0 & 0 & 0
    \end{pmatrix} \\
    \Vec{x} &= \begin{pmatrix}
        \frac{\sqrt{3}}{3} \\ \frac{\sqrt{3}}{3} \\ \frac{\sqrt{3}}{3}
    \end{pmatrix} \\
    \text{Max: }& Q(\Vec{x}) = 6
\end{align}

\subsection{Constrained Optimization and Eigenvalues}
\begin{theorem}
    If \(Q = \Vec{x}^T A \Vec{x}\), \(A\) is a real \(n\times n\) symmetric matrix, with eigenvalues:
    \[\lambda_1 \ge \lambda_2 \dots \ge \lambda_n\]
    and associated normalized eigenvectors:
    \[\Vec{u_1},\Vec{u_2},\dots,\Vec{u_n}\]
    Then, subject to the constraint \(||\Vec{x}|| = 1\),
    \begin{itemize}
        \item the maximum value of \(Q(\Vec{x}) = \lambda_1\), attained at \(\Vec{x}=\pm \Vec{u_1}\).
        \item the minimum value of \(Q(\Vec{x}) = \lambda_n\), attained at \(\Vec{x}=\pm \Vec{u_n}\).
    \end{itemize}
\end{theorem}
\section{Singular Value Decomposition}
\subsection{Introduction}
\begin{equation}
    A = U \Sigma V^T
\end{equation}

\noindent
\newline
\textbf{Facts:}
\begin{itemize}
    \item \(U\) and \(V^T\) is an orthogonal matrix.
    \item \(\Sigma\) is a diagonal matrix and the same size as \(A\).
\end{itemize}

\noindent
\newline
\textbf{Steps:}
\begin{enumerate}
    \item Compute \(A^T A\).
    \item Find eigenvalues of \(A^T A\), call them \(\sigma_i^2\).
    \item Find orthonormal eigenvectors of \(A^T A\), call them \(v_i\).
    \item Compute \(u_i = \frac{1}{\sigma_i} A v_i\)
    \item \(A = U \Sigma V^T\), where \(U = \begin{pmatrix}
        u_1 & u_2 & \dots & u_m
    \end{pmatrix}\) and \(V = \begin{pmatrix}
        v_1 & v_2 & \dots & v_n
    \end{pmatrix}\) are both orthogonal matrices.
\end{enumerate}

\begin{theorem} Singular Value Decomposition

    A \(m\times n\) matrix with rank \(r\) and non-zero singular values \(\sigma_1 \ge \sigma_2 \ge \dots \ge \sigma_r\) has a decomposition \(U \Sigma V^T\) where
    \[\Sigma = \begin{pmatrix}
        D & 0 \\ 0 & 0
    \end{pmatrix}_{m\times n} = \begin{pmatrix}
        \sigma_1 & 0 & \dots & 0 & \\
        0 & \sigma_2 & \dots & \cdot & 0 \\
        \cdot & \cdot & \ddots & \cdot & \\
        0 & 0 & \dots & \sigma_r & \\
        & 0 & & & 0
    \end{pmatrix}\]
    \(U\) is \(m\times m\) orthogonal matrix, and \(V\) is a \(n\times n\) orthogonal matrix.
\end{theorem}

\subsection{Computing the SVD}
\noindent
\newline
\textbf{Example 1: Square, Full Rank Matrix}
\begin{align}
    \begin{pmatrix}
        3 & 0 \\ 8 & 3 
    \end{pmatrix} = \begin{pmatrix}
        \Vec{u_1} & \Vec{u_2}
    \end{pmatrix} \begin{pmatrix}
        \sigma_1 & 0 \\ 0 & \sigma_2
    \end{pmatrix} \begin{pmatrix}
        \Vec{v_1} & \Vec{v_2}
    \end{pmatrix}^T
\end{align}

\noindent
Procedure:
\begin{enumerate}
    \item Compute \(A^T A\) and find \(\lambda\)'s.
    \item Find \(\Vec{v_1}, \Vec{v_2}\) orthonormal eigenvectors of \(A^T A\).
    \item Find \(\Vec{u_1}, \Vec{u_2}\) using \(\Vec{u_i} = \frac{1}{\sigma_i} A\Vec{v}_i\).
\end{enumerate}

\begin{align}
    A^T A &= \begin{pmatrix}
        3 & 8 \\ 0 & 3
    \end{pmatrix} \begin{pmatrix}
        3 & 0 \\ 8 & 3
    \end{pmatrix} = \begin{pmatrix}
        73 & 24 \\ 24 & 9
    \end{pmatrix} \\
    p(\lambda) &= \lambda^2 - 82 \lambda + 81 = 0 \\
    &= (\lambda - 81)(\lambda - 1) = 0 \\
    \lambda &= 81, 1 \\
    \sigma &= 9, 1
\end{align}

\begin{align}
    A^T A - 81 I_2 &\Rightarrow \begin{pmatrix}
        1 & -3 \\ 0 & 0
    \end{pmatrix} \;\;\; v_1 = \begin{pmatrix}
        \frac{3}{\sqrt{10}} \\ \frac{1}{\sqrt{10}}
    \end{pmatrix} \\
    A^T A - I_2 &\Rightarrow \begin{pmatrix}
        1 & \frac{1}{3} \\ 0 & 0
    \end{pmatrix} \;\;\; v_2 = \begin{pmatrix}
        -\frac{1}{\sqrt{10}} \\ \frac{3}{\sqrt{10}}
    \end{pmatrix}
\end{align}

\begin{align}
    \Vec{u_1} &= \frac{1}{9} \begin{pmatrix}
        3 & 0 \\ 8 & 3
    \end{pmatrix} \begin{pmatrix}
        \frac{3}{\sqrt{10}} \\ \frac{1}{\sqrt{10}}
    \end{pmatrix} = \begin{pmatrix}
        \frac{1}{\sqrt{10}} \\ \frac{3}{\sqrt{10}}
    \end{pmatrix} \\
    \Vec{u_2} &= \frac{1}{9} \begin{pmatrix}
        3 & 0 \\ 8 & 3
    \end{pmatrix} \begin{pmatrix}
        -\frac{1}{\sqrt{10}} \\ \frac{3}{\sqrt{10}}
    \end{pmatrix} = \begin{pmatrix}
        -\frac{3}{\sqrt{10}} \\ \frac{1}{\sqrt{10}}
    \end{pmatrix}
\end{align}

\begin{align}
    \begin{pmatrix}
        3 & 0 \\ 8 & 3 
    \end{pmatrix} = \begin{pmatrix}
        \frac{1}{\sqrt{10}} & -\frac{3}{\sqrt{10}} \\
        \frac{3}{\sqrt{10}} & \frac{1}{\sqrt{10}}
    \end{pmatrix} \begin{pmatrix}
        9 & 0 \\ 0 & 1
    \end{pmatrix} \begin{pmatrix}
        \frac{3}{\sqrt{10}} & \frac{1}{\sqrt{10}} \\
        -\frac{1}{\sqrt{10}} & \frac{3}{\sqrt{10}}
    \end{pmatrix}
\end{align}

\noindent
\newline
\textbf{Example 2: Non-square, Full Rank Matrix}
\begin{align}
    \begin{pmatrix}
        3 & -3 \\ 0 & 0 \\ 1 & 1 
    \end{pmatrix} = \begin{pmatrix}
        \Vec{u_1} & \Vec{u_2} & \Vec{u_3}
    \end{pmatrix} \begin{pmatrix}
        \sigma_1 & 0 \\ 0 & \sigma_2 \\ 0 & 0
    \end{pmatrix} \begin{pmatrix}
        \Vec{v_1} & \Vec{v_2}
    \end{pmatrix}^T
\end{align}

\noindent
Procedure:
\begin{enumerate}
    \item Compute \(A^T A\) and find \(\lambda\)'s.
    \item Find \(\Vec{v_1}, \Vec{v_2}\) orthonormal eigenvectors of \(A^T A\).
    \item Find \(\Vec{u_1}, \Vec{u_2}\) using \(\Vec{u_i} = \frac{1}{\sigma_i} A\Vec{v}_i\).
    \item Find \(\Vec{u_3}\) such that it is orthogonal to \(\Vec{u_1},\Vec{u_2}\) and is of unit length.
\end{enumerate}

\begin{align}
    A^T A &= \begin{pmatrix}
        3 & 0 & 1 \\ -3 & 0 & 1
    \end{pmatrix} \begin{pmatrix}
        3 & -3 \\ 0 & 0 \\ 1 & 1 
    \end{pmatrix} = \begin{pmatrix}
        10 & -8 \\ -8 & 10
    \end{pmatrix} \\
    p(\lambda) &= \lambda^2 - 20 \lambda + 36 = 0 \\
    &= (\lambda - 18)(\lambda - 2) = 0 \\
    \lambda &= 18, 2 \\
    \sigma &= 3\sqrt{2}, \sqrt{2}
\end{align}

\begin{align}
    A^T A - 18 I_2 &\Rightarrow \begin{pmatrix}
        1 & 1 \\ 0 & 0
    \end{pmatrix} \;\;\; v_1 = \begin{pmatrix}
        \frac{1}{\sqrt{2}} \\ -\frac{1}{\sqrt{2}}
    \end{pmatrix} \\
    A^T A - 2 I_2 &\Rightarrow \begin{pmatrix}
        1 & -1 \\ 0 & 0
    \end{pmatrix} \;\;\; v_2 = \begin{pmatrix}
        \frac{1}{\sqrt{2}} \\ \frac{1}{\sqrt{2}}
    \end{pmatrix}
\end{align}

\begin{align}
    \Vec{u_1} &= \frac{1}{3\sqrt{2}} \begin{pmatrix}
        3 & -3 \\ 0 & 0 \\ 1 & 1
    \end{pmatrix} \begin{pmatrix}
        \frac{1}{\sqrt{2}} \\ -\frac{1}{\sqrt{2}}
    \end{pmatrix} = \begin{pmatrix}
        1 \\ 0 \\ 0
    \end{pmatrix} \\
    \Vec{u_2} &= \frac{1}{\sqrt{2}} \begin{pmatrix}
        3 & 0 \\ 8 & 3
    \end{pmatrix} \begin{pmatrix}
        \frac{1}{\sqrt{2}} \\ \frac{1}{\sqrt{2}}
    \end{pmatrix} = \begin{pmatrix}
        0 \\ 0 \\ 1
    \end{pmatrix}
\end{align}

\begin{align}
    \Vec{u_3} \cdot \Vec{u_1} = 0, \;\; \Vec{u_3} \cdot \Vec{u_2} &= 0, \;\; ||\Vec{u_3}|| = 1 \;\; \\
    \Vec{u_3} &= \begin{pmatrix}
        0 \\ 1 \\ 0
    \end{pmatrix}
\end{align}

\begin{align}
    \begin{pmatrix}
        3 & 0 \\ 8 & 3 
    \end{pmatrix} = \begin{pmatrix}
        1 & 0 & 0 \\ 0 & 0 & 1 \\ 0 & 1 & 0
    \end{pmatrix} \begin{pmatrix}
        3\sqrt{2} & 0 \\ 0 & \sqrt{2}
    \end{pmatrix} \begin{pmatrix}
        \frac{1}{\sqrt{2}} & -\frac{1}{\sqrt{2}} \\
        \frac{1}{\sqrt{2}} & \frac{1}{\sqrt{2}}
    \end{pmatrix}
\end{align}

\noindent
\newline
\textbf{Example 3: Tall, Full Rank Matrix}
\begin{align}
    \begin{pmatrix}
        2 & 0 \\ 0 & -3 \\ 0 & 0 \\ 0 & 0
    \end{pmatrix} = \begin{pmatrix}
        \Vec{u_1} & \Vec{u_2} & \Vec{u_3} & \Vec{u_4}
    \end{pmatrix} \begin{pmatrix}
        \sigma_1 & 0 \\ 0 & \sigma_2 \\ 0 & 0 \\ 0 & 0
    \end{pmatrix} \begin{pmatrix}
        \Vec{v_1} & \Vec{v_2}
    \end{pmatrix}^T
\end{align}

\noindent
Procedure:
\begin{enumerate}
    \item Compute \(A^T A\) and find \(\lambda\)'s.
    \item Find \(\Vec{v_1}, \Vec{v_2}\) orthonormal eigenvectors of \(A^T A\).
    \item Find \(\Vec{u_1}, \Vec{u_2}\) using \(\Vec{u_i} = \frac{1}{\sigma_i} A\Vec{v}_i\) for nonzero \(\sigma\).
    \item Find \(\Vec{u_3}, \Vec{u_4}\) such that it forms an orthogonal basis for \(\Re^4\). Do this by finding the \(\text{Nul} \; \begin{pmatrix}
        \Vec{u_1}^T \\ \Vec{u_2}^T
    \end{pmatrix}\), since \(\text{Nul} \; \begin{pmatrix}
        \Vec{u_1}^T \\ \Vec{u_2}^T
    \end{pmatrix} = \left(\text{Row} \begin{pmatrix}
        \Vec{u_1} & \Vec{u_2}
    \end{pmatrix} \right)^\perp\).
\end{enumerate}

\begin{align}
    A^T A &= \begin{pmatrix}
        2 & 0 & 0 & 0 \\ 0 & -3 & 0 & 0
    \end{pmatrix} \begin{pmatrix}
        2 & 0 \\ 0 & -3 \\ 0 & 0 \\ 0 & 0
    \end{pmatrix} = \begin{pmatrix}
        4 & 0 \\ 0 & 9
    \end{pmatrix} \\
    \lambda &= 9, 4 \\
    \sigma &= 3, 2
\end{align}

\begin{align}
    A^T A - 9 I_2 &\Rightarrow \begin{pmatrix}
        -5 & 0 \\ 0 & 0
    \end{pmatrix} \;\;\; v_1 = \begin{pmatrix}
        0 \\ 1
    \end{pmatrix} \\
    A^T A - 4 I_2 &\Rightarrow \begin{pmatrix}
        0 & 0 \\ 0 & 5
    \end{pmatrix} \;\;\; v_1 = \begin{pmatrix}
        1 \\ 0
    \end{pmatrix}
\end{align}

\begin{align}
    \Vec{u_1} &= \frac{1}{3} \begin{pmatrix}
        2 & 0 \\ 0 & -3 \\ 0 & 0 \\ 0 & 0
    \end{pmatrix} \begin{pmatrix}
        0 \\ 1
    \end{pmatrix} = \begin{pmatrix}
        0 \\ -1 \\ 0 \\ 0
    \end{pmatrix} \\
    \Vec{u_2} &= \frac{1}{2} \begin{pmatrix}
        2 & 0 \\ 0 & -3 \\ 0 & 0 \\ 0 & 0
    \end{pmatrix} \begin{pmatrix}
        1 \\ 0
    \end{pmatrix} = \begin{pmatrix}
        1 \\ 0 \\ 0 \\ 0
    \end{pmatrix}
\end{align}

\begin{align}
    \begin{pmatrix}
        \Vec{u_1}^T \\ \Vec{u_2}^T
    \end{pmatrix} &= \begin{pmatrix}
        0 & -1 & 0 & 0 \\ 1 & 0 & 0 & 0
    \end{pmatrix} \Rightarrow \begin{pmatrix}
        1 & 0 & 0 & 0 \\ 0 & 1 & 0 & 0
    \end{pmatrix} \\
    \Vec{u_3} &= \begin{pmatrix}
        0 \\ 0 \\ 1 \\ 0
    \end{pmatrix}, \; \Vec{u_4} = \begin{pmatrix}
        0 \\ 0 \\ 0 \\ 1
    \end{pmatrix}
\end{align} 

\begin{align}
    \begin{pmatrix}
        3 & 0 \\ 8 & 3 
    \end{pmatrix} = \begin{pmatrix}
        0 & 1 & 0 & 0 \\
        -1 & 0 & 0 & 0 \\
        0 & 0 & 1 & 0 \\
        0 & 0 & 0 & 1 
    \end{pmatrix} \begin{pmatrix}
        3 & 0 \\ 0 & 2 \\ 0 & 0 \\ 0 & 0
    \end{pmatrix} \begin{pmatrix}
        0 & 1 \\ 1 & 0
    \end{pmatrix}
\end{align}

\noindent
\newline
\textbf{Example 4: Tall, Non-Full Rank Matrix}
\begin{align}
    \begin{pmatrix}
        1 & -1 \\ -2 & 2 \\ 2 & -2
    \end{pmatrix} = \begin{pmatrix}
        \Vec{u_1} & \Vec{u_2} & \Vec{u_3}
    \end{pmatrix} \begin{pmatrix}
        \sigma_1 & 0 \\ 0 & 0 \\ 0 & 0
    \end{pmatrix} \begin{pmatrix}
        \Vec{v_1} & \Vec{v_2}
    \end{pmatrix}^T
\end{align}

\noindent
Procedure:
\begin{enumerate}
    \item Compute \(A^T A\) and find \(\lambda\)'s.
    \item Find \(\Vec{v_1}, \Vec{v_2}\) orthonormal eigenvectors of \(A^T A\).
    \item Find \(\Vec{u_1}\) using \(\Vec{u_i} = \frac{1}{\sigma_i} A\Vec{v}_i\) for nonzero \(\sigma\).
    \item Find \(\Vec{u_2}, \Vec{u_3}\) such that it forms an orthogonal basis for \(\Re^4\). One way to do this is by finding the \(\text{Nul} \; \begin{pmatrix}
        \Vec{u_1}^T
    \end{pmatrix}\), since \(\text{Nul} \; \begin{pmatrix}
        \Vec{u_1}^T
    \end{pmatrix} = \left(\text{Row} \begin{pmatrix}
        \Vec{u_1}
    \end{pmatrix} \right)^\perp\), and then performing the Gram-Schmidt algorithm on the null space basis to get an orthonormal basis. Another way to do this is by performing the Gram-Schmidt algorithm on \(\{ \Vec{u_1}, \Vec{e_1}, \Vec{e_2} \}\).
\end{enumerate}

\begin{align}
    A^T A &= \begin{pmatrix}
        1 & -2 & 2 \\ -1 & 2 & -2
    \end{pmatrix} \begin{pmatrix}
        1 & -1 \\ -2 & 2 \\ 2 & -2
    \end{pmatrix} = \begin{pmatrix}
        9 & -9 \\ -9 & 9
    \end{pmatrix} \\
    p(\lambda) &= \lambda^2 - 18 \lambda = 0 \\
    &= \lambda (\lambda - 18) = 0 \\
    \lambda &= 18, 0 \\
    \sigma &= 3\sqrt{2}, 0
\end{align}

\begin{align}
    A^T A - 18 I_2 &\Rightarrow \begin{pmatrix}
        1 & 1 \\ 0 & 0
    \end{pmatrix} \;\;\; v_1 = \begin{pmatrix}
        -\frac{1}{\sqrt{2}} \\ \frac{1}{\sqrt{2}}
    \end{pmatrix} \\
    A^T A - 0 I_2 &\Rightarrow \begin{pmatrix}
        1 & -1 \\ 0 & 0
    \end{pmatrix} \;\;\; v_2 = \begin{pmatrix}
        \frac{1}{\sqrt{2}} \\ \frac{1}{\sqrt{2}}
    \end{pmatrix}
\end{align}

\begin{align}
    \Vec{u_1} &= \frac{1}{3\sqrt{2}} \begin{pmatrix}
        1 & -1 \\ -2 & 2 \\ 2 & -2
    \end{pmatrix} \begin{pmatrix}
        -\frac{1}{\sqrt{2}} \\ \frac{1}{\sqrt{2}}
    \end{pmatrix} = \begin{pmatrix}
        -\frac{1}{3} \\ \frac{2}{3} \\ -\frac{2}{3}
    \end{pmatrix}
\end{align}

\begin{align}
    \begin{pmatrix}
        \Vec{u_1}^T
    \end{pmatrix} &\Rightarrow \begin{pmatrix}
        1 & -2 & 2
    \end{pmatrix}  \\ 
    \Vec{x} &= r \begin{pmatrix}
        2 \\ 1 \\ 0
    \end{pmatrix} + s \begin{pmatrix}
        -2 \\ 0 \\ 1
    \end{pmatrix} \\
\end{align}

\noindent
Performing the Gram-Schmidt Algorithm on \(\left\{ \begin{pmatrix}
    2 \\ 1 \\ 0
\end{pmatrix}, \begin{pmatrix}
    -2 \\ 0 \\ 1
\end{pmatrix}\right\}\),

\begin{align}
    \Vec{y_1} &= \Vec{x_1} = \begin{pmatrix}
        2 \\ 1 \\ 0
    \end{pmatrix} \\
    \Vec{y_2} &= \Vec{x_2} - \frac{\Vec{x_2} \cdot \Vec{y_1}}{\Vec{y_1} \cdot \Vec{y_1}} \Vec{y_1} = \begin{pmatrix}
        -2 \\ 4 \\ 5
    \end{pmatrix} \\
    \Vec{u_1} &= \begin{pmatrix}
        \frac{2}{\sqrt{5}} \\ \frac{1}{\sqrt{5}} \\ 0
    \end{pmatrix} \\
    \Vec{u_2} &= \begin{pmatrix}
        -\frac{2}{\sqrt{45}} \\ \frac{4}{\sqrt{45}} \\ \frac{5}{\sqrt{45}}
    \end{pmatrix}
\end{align}

\begin{align}
    \begin{pmatrix}
        1 & -2 & 2 \\ -1 & 2 & -2
    \end{pmatrix} = \begin{pmatrix}
        -\frac{1}{3} & \frac{2}{\sqrt{5}} & -\frac{2}{\sqrt{45}} \\ \frac{2}{3} & \frac{1}{\sqrt{5}} & \frac{4}{\sqrt{45}} \\ -\frac{2}{3} & 0 & \frac{5}{\sqrt{45}}
    \end{pmatrix}
    \begin{pmatrix}
        3\sqrt{2} & 0 \\ 0 & 0 \\ 0 & 0
    \end{pmatrix} \begin{pmatrix}
        \frac{1}{\sqrt{2}} & -\frac{1}{\sqrt{2}} \\
        \frac{1}{\sqrt{2}} & \frac{1}{\sqrt{2}}
    \end{pmatrix}
\end{align}

\noindent
\newline
\textbf{Example 5: Wide, Full Rank Matrix}
\begin{align}
    \begin{pmatrix}
        3 & 0 & 1 \\ -3 & 0 & 1
    \end{pmatrix} = \begin{pmatrix}
        \Vec{u_1} & \Vec{u_2}
    \end{pmatrix} \begin{pmatrix}
        \sigma_1 & 0 & 0 \\ 0 & \sigma_2 & 0
    \end{pmatrix} \begin{pmatrix}
        \Vec{v_1} & \Vec{v_2} & \Vec{v_3}
    \end{pmatrix}^T
\end{align}

\noindent
Procedure:
\begin{enumerate}
    \item Since finding the SVD of a wide matrix is more computationally intensive, one can find the SVD of \(B=A^T\) and then apply \(A = (U \Sigma V^T)^T = V \Sigma^T U^T\).
    \item Compute \(B^T B\) and find \(\lambda\)'s.
    \item Find \(\Vec{v_1}, \Vec{v_2}\) orthonormal eigenvectors of \(B^T B\).
    \item Find \(\Vec{u_1}, \Vec{u_2}\) using \(\Vec{u_i} = \frac{1}{\sigma_i} A\Vec{v}_i\) for nonzero \(\sigma\).
    \item Find \(\Vec{u_3}\) such that it forms an orthogonal basis for \(\Re^3\).
\end{enumerate}

\begin{align}
    B^T B &= \begin{pmatrix}
        3 & 0 & 1 \\ -3 & 0 & 1
    \end{pmatrix} \begin{pmatrix}
        3 & -3 \\ 0 & 0 \\ 1 & 1
    \end{pmatrix} = \begin{pmatrix}
        10 & -8 \\ -8 & 10
    \end{pmatrix} \\
    p(\lambda) &= \lambda^2 - 20 \lambda + 36 = 0 \\
    &= (\lambda - 18)(\lambda - 2) = 0 \\
    \lambda &= 18, 2, 0 \\
    \sigma &= 3\sqrt{2}, \sqrt{2}
\end{align}

\begin{align}
    B^T B - 18 I_2 &\Rightarrow \begin{pmatrix}
        1 & 1 \\ 0 & 0
    \end{pmatrix} \;\;\; v_1 = \begin{pmatrix}
        \frac{1}{\sqrt{2}} \\ -\frac{1}{\sqrt{2}}
    \end{pmatrix} \\
    B^T B - 2 I_2 &\Rightarrow \begin{pmatrix}
        1 & -1 \\ 0 & 0
    \end{pmatrix} \;\;\; v_2 = \begin{pmatrix}
        \frac{1}{\sqrt{2}} \\ \frac{1}{\sqrt{2}}
    \end{pmatrix}
\end{align}

\begin{align}
    \Vec{u_1} &= \frac{1}{3\sqrt{2}} \begin{pmatrix}
        3 & -3 \\ 0 & 0 \\ 1 & 1
    \end{pmatrix} \begin{pmatrix}
        \frac{1}{\sqrt{2}} \\ -\frac{1}{\sqrt{2}}
    \end{pmatrix} = \begin{pmatrix}
        1 \\ 0 \\ 0
    \end{pmatrix} \\ 
    \Vec{u_2} &= \frac{1}{\sqrt{2}} \begin{pmatrix}
        3 & -3 \\ 0 & 0 \\ 1 & 1
    \end{pmatrix} \begin{pmatrix}
        \frac{1}{\sqrt{2}} \\ \frac{1}{\sqrt{2}}
    \end{pmatrix} = \begin{pmatrix}
        0 \\ 0 \\ 1
    \end{pmatrix}
\end{align}

\begin{align}
    \Vec{u_3} &= \Vec{u_1} \times \Vec{u_2} = \begin{pmatrix}
        0 \\ 1 \\ 0
    \end{pmatrix}
\end{align}

\begin{align}
    A^T &= B = \begin{pmatrix}
        3 & -3 \\ 0 & 0 \\ 1 & 1
    \end{pmatrix} = \begin{pmatrix}
        1 & 0 & 0 \\ 0 & 0 & 1 \\ 0 & 1 & 0
    \end{pmatrix}
    \begin{pmatrix}
        3\sqrt{2} & 0 \\ 0 & \sqrt{2} \\ 0 & 0
    \end{pmatrix} \begin{pmatrix}
        \frac{1}{\sqrt{2}} & -\frac{1}{\sqrt{2}} \\
        \frac{1}{\sqrt{2}} & \frac{1}{\sqrt{2}}
    \end{pmatrix} \\
    A &= V \Sigma^T U^T = \begin{pmatrix}
        3 & 0 & 1 \\ -3 & 0 & 1
    \end{pmatrix} = \begin{pmatrix}
        \frac{1}{\sqrt{2}} & \frac{1}{\sqrt{2}} \\
        -\frac{1}{\sqrt{2}} & \frac{1}{\sqrt{2}}
    \end{pmatrix} \begin{pmatrix}
        3\sqrt{2} & 0 & 0 \\ 0 & \sqrt{2} & 0
    \end{pmatrix} \begin{pmatrix}
        1 & 0 & 0 \\ 0 & 0 & 1 \\ 0 & 1 & 0
    \end{pmatrix}
\end{align}

\subsection{Low Rank Approximation of \(A\) using SVD}
\begin{theorem} Spectral Decomposition Theorem

    Since the singular values in the \(\Sigma\) matrix are ordered from largest to smallest, the following can be used to derive the n-rank approximation of \(A_n\).
    \begin{align}
        A_n = \sum_{i=1}^n \sigma_i \Vec{u_i} \Vec{v_i}^T
    \end{align}
\end{theorem}

\noindent
\newline
\textbf{Example 1: Compute the spectral decomposition.}
\begin{align}
    A &= \begin{pmatrix}
        3 & -3 \\ 0 & 0 \\ 1 & 1
    \end{pmatrix} = \begin{pmatrix}
        1 & 0 & 0 \\ 0 & 0 & 1 \\ 0 & 1 & 0
    \end{pmatrix}
    \begin{pmatrix}
        3\sqrt{2} & 0 \\ 0 & \sqrt{2} \\ 0 & 0
    \end{pmatrix} \begin{pmatrix}
        \frac{1}{\sqrt{2}} & -\frac{1}{\sqrt{2}} \\
        \frac{1}{\sqrt{2}} & \frac{1}{\sqrt{2}}
    \end{pmatrix} \\
    &= 3\sqrt{2} \begin{pmatrix}
        1 \\ 0 \\ 0
    \end{pmatrix} \begin{pmatrix}
        \frac{1}{\sqrt{2}} & -\frac{1}{\sqrt{2}}
    \end{pmatrix}
    + \sqrt{2} \begin{pmatrix}
        0 \\ 0 \\ 1
    \end{pmatrix} \begin{pmatrix}
        \frac{1}{\sqrt{2}} & \frac{1}{\sqrt{2}}
    \end{pmatrix} \\
    &= 3 \sqrt{2} \begin{pmatrix}
        \frac{1}{\sqrt{2}} & -\frac{1}{\sqrt{2}} \\ 0 & 0 \\ 0 & 0
    \end{pmatrix} + \sqrt{2} \begin{pmatrix}
        0 & 0 \\ 0 & 0 \\ \frac{1}{\sqrt{2}} & \frac{1}{\sqrt{2}}
    \end{pmatrix}
\end{align}

\subsection{Singular Values}
\begin{definition} Singular Values

    The singular values of \(A\) are defined to be:
    \[\sigma_1 = \sqrt{\lambda_1} \ge \sigma_1 = \sqrt{\lambda_2} \dots \ge \sigma_n = \sqrt{\lambda_n}\]
\end{definition}

\noindent
The matrix \(A^T A\) is always symmetric, with non-negative eigenvalues \(\lambda_1 \ge \lambda_2 \ge \dots \ge \lambda_n \ge 0\). Let \(\{ \Vec{v_1}, \dots, \Vec{v_n} \}\) be the associated orthonormal eigenvectors. Then
\begin{align}
   ||A \Vec{v_j}||^2 &= (A \Vec{v_j}) \cdot (A \Vec{v_j}) \\
   &= \Vec{v_j}^T A^T A \Vec{v_j} \\
   &= \lambda_j \Vec{v_j} \cdot \Vec{v_j} \\
   &= \lambda_j ||\Vec{v_j}||^2 \\
   &= \lambda_j
\end{align}

\noindent
If \(A\) has rank \(r\), then \(\{ A \Vec{v_1}, \dots, A \Vec{v_r}\}\) is an orthogonal basis for \(\text{Col} A\). For \(1\le j < k \le r\)
\begin{align}
    (A \Vec{v_j})^T A \Vec{v_k} &= \Vec{v_j}^T A^T A \Vec{v_k} \\
    &= \lambda_k \Vec{v_j}^T \Vec{v_k} \\
    &= \lambda_k \Vec{v_j} \cdot \Vec{v_k} = 0
\end{align}

\subsection{SVD and the Four Subspaces}
Suppose the \(\text{rank} \; A\) is \(r\). Given the SVD of \(A\),
\begin{equation}
    A = U \Sigma V^T
\end{equation}

\noindent
An orthonormal basis for the \(\text{Col}\; A\) are the first \(r\) columns of \(U\). The remaining columns of \(U\) form an orthonormal basis for the \(\text{Nul}\; A^T\). An orthonormal basis for the \(\text{Row}\; A\) are the first \(r\) columns of \(V\). Thus, an orthonormal basis for the \(\text{Row}\; A\) are the first \(r\) rows of \(V^T\). The remaining rows of \(V^T\) form an orthonormal basis for the \(\text{Nul}\; A\).

\noindent
\newline
In other words,
\begin{enumerate}
    \item \(A \Vec{v_s} = \sigma_s \Vec{u_s}\)
    \item \(\Vec{v_1}, \dots, \Vec{v_r}\) is an orthonormal basis for \(\text{Row} \; A\).
    \item \(\Vec{u_1}, \dots, \Vec{u_r}\) is an orthonormal basis for \(\text{Col} \; A\).
    \item \(\Vec{v_{r+1}}, \dots, \Vec{v_n}\) is an orthonormal basis for \(\text{Nul} \; A\).
    \item \(\Vec{u_{r+1}}, \dots, \Vec{u_n}\) is an orthonormal basis for \(\text{Nul} \; A^T\).
\end{enumerate}

\subsection{Geometric Interpretation of SVD}
Since \(U\) and \(V^T\) are orthogonal, both matrices are pure rotations. Since \(\Sigma\) is a diagonal matrix, it is a pure dilation along the elementary basis vector directions.
\begin{figure}[H]
    \centering
    \includegraphics[width=60mm,height=\textheight,keepaspectratio]{images/svd_transform.png}
    \caption{Geometric Interpretation of SVD}
    \label{fig:svd_geometric}
\end{figure}

\subsection{Algorithm to find SVD of \(A\)}
Suppose \(A\) is \(m\times n\) and has rank \(r \le n\).
\begin{enumerate}
    \item Compute the squared singular values \(\sigma_i^2\) of \(A^T A\) and construct \(\Sigma\). This is done by finding the eigenvalues of \(A^T A\), call then \(\lambda_i \ge 0\) and set \(\sigma_i = \sqrt{\lambda_i}\).
    \item Compute the unit singular vectors \(\Vec{v_i}\) of \(A^T A\) and use them to form \(V\). Since \(A^T A\) is symmetric, by the spectral theorem, the algebraic multiplicity is equal to the geometric multiplicity and we can find orthonormal basis \(\{ \Vec{v_1},\dots,\Vec{v_n} \}\) of \(\Re^n\) constituting of eigenvectors of \(A^T A\).
    \item Compute an orthonormal basis for \(\text{Col} \; A\) using the following for any \(\sigma_i \ne 0\):
    \[\Vec{u_i} = \frac{1}{\sigma_i} A \Vec{v_i}, \;\;\; i = 1,2,\dots,r\]
    Extend the set \(\{ \Vec{u_i} \}\) to form an orthogonal basis for \(\Re^m\), use the basis for form \(U\).
\end{enumerate}

\subsection{Condition Number of a Matrix}
\begin{definition} Condition Number

    If \(A\) is invertible \(n \times n\) matrix, the ratio:
    \[\frac{\sigma_1}{\sigma_n}\]
    is the condition number of \(A\).

    Note that:
    \begin{itemize}
        \item The condition number of a matrix describes the sensitivity of a solution to \(A \Vec{x} = \Vec{b}\) is to errors in \(A\).
        \item We could define the condition number for a rectangular matrix, but that would go beyond the scope of this course.
    \end{itemize}
\end{definition}

\end{document}
