\section{Matrix Factorization}
Examples:
\begin{align}
    A &= LU \\
    A &= QR \\
    A &= PDP^{-1} \\
    A &= U \Sigma V^T
\end{align}

\subsection{LU Decomposition}
\begin{theorem}
    If \(A\) is an \(m \times n\) matrix that can be row reduced to echelon form without row exchanges, then \(A = LU\). \(L\) is a lower triangular \(m \times m\) matrix with 1's on the diagonal. \(U\) is an echelon form of \(A\).
\end{theorem}

\begin{definition} Types of Matrices
    \newline
    Upper Triangular Matrix: if \(a_{i,j} = 0\) for \(i > j\).
    \newline
    Lower Triangular Matrix: if \(a_{i,j} = 0\) for \(i < j\).
    \newline
    Diagonal: Both upper and lower, if \(a_{i,j} = 0\) for \(i \ne j\).
\end{definition}

\noindent
\newline
Example 1:
\begin{enumerate}
    \item \(4R_1 + R_2 \rightarrow R_2\)
    \item \(-2R_1 + R_3 \rightarrow R_3\)
    \item \(-5R_1 + R_3 \rightarrow R_3\)
\end{enumerate}

\begin{align}
    A &= \begin{pmatrix}
        4 & -3 & -1 & 5 \\
        -16 & 12 & 2 & -17 \\
        8 & -6 & -12 & 22
    \end{pmatrix} \\
    &\begin{pmatrix}
        4 & -3 & -1 & 5 \\
        0 & 0 & -2 & 3 \\
        0 & 0 & -10 & 12
    \end{pmatrix} \\
    U &= \begin{pmatrix}
        4 & -3 & -1 & 5 \\
        0 & 0 & -2 & 3 \\
        0 & 0 & 0 & -3
    \end{pmatrix}
\end{align}

\noindent
Use the scaling factors to determine \(L\),
\begin{align}
    L = \begin{pmatrix}
        1 & 0 & 0 \\
        -4  & 1 & 0 \\
        2 & 5 & 1
    \end{pmatrix}
\end{align}

\subsection{Why LU Decomposition works?}
\begin{align}
    E_3 E_2 E_1 A &= U \\
    A &= E_1^{-1} E_2^{-1} E_3^{-1} U \\
    A &= L U
\end{align}

\noindent
\newline
LU decompositions do not always exist and the decompositions are not always unique.

\subsection{Solving using LU Decomposition}
\begin{enumerate}
    \item \(L\Vec{y}=\Vec{b}\)
    \item \(U\Vec{x}=\Vec{y}\)
\end{enumerate}

\noindent
\newline
Example:
\begin{align}
    A &= LU = \begin{pmatrix}
        1 & 0 & 0 & 0 \\
        1 & 1 & 0 & 0 \\
        1 & 2 & 1 & 0 \\
        0 & -1 & -1 & 1
    \end{pmatrix}\begin{pmatrix}
        1 & 1 & 0 \\
        0 & -1 & -1 \\
        0 & 0 & 2 \\
        0 & 0 & 0
    \end{pmatrix} \\
    \Vec{b} &= \begin{pmatrix}
        16 \\ 2 \\ -4 \\6
    \end{pmatrix}
\end{align}

\noindent
Step 1: Row reduce \((L|\Vec{b})\).
\begin{equation}
    \begin{pmatrix}[cccc|c]
        1 & 0 & 0 & 0 & 16 \\
        0 & 1 & 0 & 0 & -14 \\
        0 & 0 & 1 & 0 & 8 \\
        0 & 0 & 0 & 1 & 0
    \end{pmatrix}
\end{equation}

\noindent
Step 2: Row reduce \((U|\Vec{y})\).
\begin{equation}
    \begin{pmatrix}[ccc|c]
        1 & 0 & 0 & 6 \\
        0 & 1 & 0 & 10 \\
        0 & 0 & 1 & 4 \\
        0 & 0 & 0 & 0
    \end{pmatrix}
\end{equation}

\noindent
\newline
Therefore, \(\Vec{x} = \begin{pmatrix}
    6 \\ 10 \\ 4
\end{pmatrix}\).